Wie im vorherigen Kapitel angesprochen, ist das Ziel der Diplomarbeit,
eine App zu entwickeln, mit der man in der Lage ist, eine Inventur
durchzuführen. Doch wieso eine App und wieso überhaupt Android? Um diese
Frage zu klären, muss man zwischen zwei Begriffen unterscheiden
\cite{native-vs-web}:

\begin{itemize}
\tightlist
\item
  Native App
\item
  Web-App
\end{itemize}

\hypertarget{nativ-vs.-web}{%
\section{Nativ vs.~Web}\label{nativ-vs.-web}}

Unter einer nativen App versteht man eine App, die für ein bestimmtes
Betriebssystem geschrieben wurde \cite{native-definition}. Die
Definition ist allerdings nicht ganz eindeutig, da Frameworks wie
Flutter und Xamarin nativer Funktionalität sehr nahekommen, obwohl sie
mehrere verschiedene Betriebssysteme unterstützten. Eine Web-App
hingegen basiert auf HTML und wird per Browser aufgerufen. Sie stellt
nichts anderes als eine für mobile Geräte optimierte Website da.

Am Markt zeichnet sich in letzter Zeit ein klarer Trend ab -- native
Apps sterben allmählich aus \cite{native-trend} und werden durch mobile
Webseiten ersetzt. Das ist dadurch erklärbar, dass mobile Webseiten
immer funktionsreicher werden. Mittlerweile haben Web-Apps nur noch
geringfügig weniger Möglichkeiten als native Apps. Aus wirtschaftlicher
Betrachtungsweise amortisieren sich Web-Apps de facto um einiges
schneller und sind auch dementsprechend lukrativer.

\hypertarget{begruxfcndung-native-app}{%
\section{Begründung: Native App}\label{begruxfcndung-native-app}}

Das Projektteam hat sich dennoch für eine native App entschieden. Um
diese Entscheidung nachvollziehen zu können, ist ein tieferer Einblick
in den gegebenen Use-Case erforderlich.

Das Ziel ist es nicht, möglichst viele Downloads im Play Store zu
erzielen oder etwaige Marketingmaßnahmen zu setzen. Es soll stattdessen
mit den gegebenen Ressourcen eine Inventurlösung entwickelt werden, die
die bestmögliche Lösung für unsere Schule darstellt. Eine native App
wird eine Web-App immer hinsichtlich Qualität und User Experience klar
übertreffen. Im vorliegenden Fall wäre es sicherlich möglich, eine
Inventur mittels Web-App durchzuführen, allerdings würde diese vor allem
in den Bereichen Performanz und Verlässlichkeit Mängel aufweisen. Diese
zwei Bereiche stellen genau die zwei Problembereiche da, die es mit der
vorliegenden Gesamtlösung bestmöglich zu optimieren gilt. Des Weiteren
bieten sich native Apps ebenfalls für komplexe Projekte an, da Web-Apps
aktuell noch nicht in der Lage sind, komplexe Aufgabenstellungen mit
vergleichbar geringem Aufwand zu inkorporieren. Ein typisches Beispiel
für eine Web-App stellt eine mobile Website einer Tageszeitung
(eventuell auch mit Kommentaren, Bewertungssystemen etc.) da. Unter
Berücksichtigung dieser Gesichtspunkte wurde also der Entschluss
gefasst, eine native Applikation zu entwickeln, da diese ein insgesamt
besseres Produkt darstellen wird. Es sei gesagt, dass es auch Hybride
Apps gibt. Diese sind jedoch einer nativen App in denselben Aspekten wie
eine Web-App klar unterlegen.

\hypertarget{auswahl-der-nativen-technologie}{%
\subsection{Auswahl der nativen
Technologie}\label{auswahl-der-nativen-technologie}}

Nun gilt es zu klären, warum die App für natives Android (in Java)
entwickelt wurde. Folgende nativen Alternativen galt es abzuwägen:

\begin{itemize}
\tightlist
\item
  Flutter \cite{flutter}
\item
  Xamarin \cite{xamarin}
\item
  Native IOS
\item
  Native Android (Java/Kotlin)
\end{itemize}

Flutter ist ein von Google entwickeltes Framework, dass eine gemeinsame
Codebasis für Android und IOS anbietet. Eine gemeinsame Codebasis wird
oftmals unter dem Begriff \emph{cross-platform} zusammengefasst und
bedeutet, dass man eine mit Flutter entwickelte native App sowohl mit
Android-Geräten als auch mit IOS-Geräten verwenden kann. Flutter ist
eine relativ neue Plattform -- das erste stabile Release wurde erst im
Dezember 2018 veröffentlicht \cite{flutter-stable}. Außerdem verwendet
Flutter die Programmiersprache \emph{Dart}, die Java ähnelt. Diese
Umstände sind ein Segen und Fluch zugleich. Flutter wird in Zukunft
sicherlich weiterhin an Popularität zulegen, allerdings ist die Anzahl
an verfügbarer Dokumentation für das junge Flutter im Vergleich zu den
anderen Optionen immer noch weitaus geringer. Xamarin ähnelt Flutter in
den soeben aufgezählten Aspekten stark. Es ist ebenfalls ein
cross-platform Framework, das jedoch in C\# geschrieben wird.

Native IOS wird nur der Vollständigkeit Halber aufgelistet, stellte
allerdings zu keinem Zeitpunkt eine wirkliche Alternative da, weil
IOS-Geräte einige Eigenschaften besitzen, die für eine Inventur nicht
optimal sind (Sprichwort: Akkukapazität). Außerdem haben in etwa nur
20\% aller Geräte \cite{ios-market-share} IOS als Betriebssystem und die
Entwicklung einer IOS-App wird durch strenge Voraussetzungen äußerst
unattraktiv gemacht. So kann man beispielsweise nur auf einem
Apple-Gerät IOS-Apps entwickeln. All dies hat zum Entschluss geführt,
IOS aus dem Spiel zu lassen und uns auf Android zu fokussieren.

\hypertarget{begruxfcndung-natives-android-java}{%
\subsubsection{Begründung: Natives Android
(Java)}\label{begruxfcndung-natives-android-java}}

Die Entscheidung ist schlussendlich auf natives Android (Java) gefallen.
Es mag zwar vielleicht nicht die innovativste Entscheidung sein, stellt
aber aus folgenden Gründen die bewährteste und risikoloseste Option da:

\begin{itemize}
\tightlist
\item
  Natives Android ist eine allbekannte und weit etablierte Lösung. Die
  Wahrscheinlichkeit, dass die Unterstützung durch Google eingestellt
  wird, ist also äußerst gering.
\item
  Die App wird in den nächsten Jahren immer noch am Stand der Technik
  sein.
\item
  Natives Android hat mit großem Abstand die größte Dokumentation.
\item
  An der Schule wird Java unterrichtet. Das macht somit eventuelle
  Modifikationen nach Projektabschluss durch andere Schüler viel
  einfacher möglich.
\item
  Dadurch, dass Kotlin erst seit 2019 \cite{kotlin-preference} offiziell
  die von Google bevorzugte Sprache ist, sind die meisten Tutorials
  immer noch in Java.
\item
  Sehr viele Unternehmen haben viele aktive Java-Entwickler. Dadurch
  wird die App attraktiver, da die Unternehmensmitarbeiter (von z.B.
  allegro) keine neue Sprache lernen müssen, um Anpassungen
  durchzuführen.
\item
  Das Projektteam hat im Rahmen eines Praktikums bereits Erfahrungen mit
  nativem Java gesammelt.
\end{itemize}

Schlussendlich muss noch die Frage der unterstützten Android-Versionen
geklärt werden. Das minimale API-Level der App ist 21 - auch bekannt als
Android 5.0 `Lollipop'. Android 4.0 hat sehr viele nützliche Libraries
hervorgebracht. So zum Beispiel die \emph{Mobile Vision API} von Google,
dank derer man in der Lage ist, Barcodes in akzeptabler Zeit mit der
Kamera des Geräts zu scannen. Die Wahl ist auf 5.0 gefallen, da somit
ein Puffer zur Verfügung steht und in etwa 90\% aller Android-Geräte
ohnehin auf 5.0 oder einer neueren Version laufen
\cite{android-versions-market-share}.

\hypertarget{einfuxfchrung-zu-nativem-java}{%
\section{Einführung zu nativem
Java}\label{einfuxfchrung-zu-nativem-java}}

Um eine Basis für die folgenden Kapitel zu schaffen, werden hier die
Basics der Android-Entwicklung mit nativem Java näher beschrieben. Das
Layout einer App wird in XML Files gespeichert, währenddessen das
wirkliche Programmieren mit Java erfolgt.

\hypertarget{single-activity-app}{%
\subsection{Single-Activity-App}\label{single-activity-app}}

Als Einstiegspunkt in eine App dient eine sogenannte \emph{Activity}.
Eine Activity ist eine normale Java-Klasse, der durch Vererbung
UI-Funktionen verleiht werden.

Bis vor kurzem war es üblich, dass eine App mehrere Activities hat. Das
wird bei den Benutzern dadurch bemerkbar, dass die App z.B. bei einem
Tastendruck ein weiteres Fenster öffnet, das das bisherige überdeckt.
Das neue Fenster ist eine eigene Activity. Google hat sich nun offiziell
für sogenannte Single-Activities ausgesprochen \cite{single-activity}.
Das heißt, dass es nur eine Activity und mehrere \emph{Fragments} gibt.
Ein Fragment ist eine Teilmenge des UIs bzw. einer Activity. Anstatt
jetzt beim Tastendruck eine neue Activity zu starten, wird einfach das
aktuelle Fragment ausgetauscht. Dadurch, dass keine neuen Fenster
geöffnet werden, ist die User Expierence (UX) um ein Vielfaches besser
-- die Performanz leidet nur minimal darunter. Die vorliegende App ist
aus diesen Gründen ebenfalls eine Single-Activity-App.

\hypertarget{seperation-of-concerns}{%
\subsection{Seperation of Concerns}\label{seperation-of-concerns}}

In Android ist es eine äußerst schlechte Idee, sämtliche Logik in einer
Activity oder einem Fragment zu implementieren. Das softwaretechnische
Prinzip \emph{Seperation of Concerns (SoP)} hat unter Android einen
besonderen Stellenwert. Dieses Prinzip beschreibt im Wesentlichen, dass
eine Klasse nur einer Aufgabe dienen sollte. Falls eine Klasse mehrere
Aufgaben erfüllt, so gilt es diese auf mehrere logische Komponenten
aufzuteilen. Beispiel: Eine Activity hat immer die Verantwortung, die
Kommunikation zwischen UI und Benutzer abzuwickeln. Bad Practice wäre
es, wenn jene Activity ebenfalls dafür verantwortlich ist, Daten von
einem Server abzurufen. Das Prinzip verfolgt das Ziel, die
\emph{God Activity Architecture (GAA)} möglichst zu vermeiden
\cite{god-activities}. Eine God-Activity ist unter Android eine
Activity, die die komplette Business-Logic beinhaltet und SoP in
jeglicher Hinsicht widerspricht. God-Activities gilt es dringlichst zu
vermeiden, da sie folgende Nachteile mit sich bringen:

\begin{itemize}
\tightlist
\item
  Refactoring wird kompliziert
\item
  Wartung und Dokumentierung wird äußerst schwierig
\item
  Automatisiertes Testing (z.B. Unit-Testing) wird nahezu unmöglich
  gemacht
\item
  Größere Bug-Anfälligkeit
\item
  Im Bezug auf Android gibt es oftmals massive Probleme mit dem
  \emph{Lifecycle} einer Activity - da eine Activity und ihre Daten
  schnell vernichtet werden können (z.B. wenn der Benutzer sein Gerät
  rotiert und das Gerät den Bildschirmmodus wechselt)
\end{itemize}

God-Activities sind ein typisches Beispiel für Spaghetticode. Es bedarf
also einer wohlüberlegten und strukturierten Architektur, um diese
Probleme zu unterbinden. Im nächsten Kapitel wird dementsprechend die
Architektur der App im Detail erklärt.
