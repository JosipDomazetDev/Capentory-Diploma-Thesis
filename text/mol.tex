\hypertarget{technische-umsetzung-server}{%
\section{Technische Umsetzung:
Server}\label{technische-umsetzung-server}}

Die Inventur- sowie Gegenstandsdaten der HTL Rennweg sollen an einem
zentralen Ort verwaltet und geführt werden. Für den dafür eingesetzten
Server ergeben sich also bestimmte Anforderungen:

\begin{itemize}
\tightlist
\item
  eine einfache Datenbankverwaltung und -verbindung
\item
  das Führen einer Historie aller Zustände der Inventar-Gegenständen
\item
  eine Grundlage für eine Web-Administrationsoberfläche
\item
  die Möglichkeit für Daten Import- und Export, etwa als
  \emph{.xlsx}\index{.xlsx: Format einer Excel Datei} Datei
\item
  eine Grundlage für die Kommunikation mit der Client-Applikation
\item
  hohe Stabilität und Verfügbarkeit
\end{itemize}

Angesichts der Programmiersprachen, in die das Projektteam spezialisiert
ist, stehen als Backend-Lösung 4
\emph{Frameworks}\index{Framework: Eine softwaretechnische Architektur, die bestimmte Funktionen und Klassen zur Verfügung stellt.}
zur öffentlichen Verfügung, zwischen denen entschieden wurde:

\begin{itemize}
\tightlist
\item
  Django \cite{django}
\item
  Pyramid \cite{pyramid}
\item
  Web2Py \cite{web2py}
\item
  Flask \cite{flask}
\end{itemize}

Alle genannten Alternativen sind
\emph{Frameworks}\index{Framework: Eine softwaretechnische Architektur, die bestimmte Funktionen und Klassen zur Verfügung stellt.}
der Programmiersprache Python. Gewählt wurde die Alternative ``Django''
aufgrund einer bestehenden und frei verfügbaren
Inventarverwaltungsplattform ``Ralph'' \cite{ralph}, welche auf Django
aufbaut und durch die vorliegende Diplomarbeit hinreichend erweitert
wird.

\hypertarget{django-und-ralph}{%
\subsection{Django und Ralph}\label{django-und-ralph}}

Django ist ein in der Programmiersprache Python geschriebenes
Webserver-\emph{Framework}\index{Framework: Eine softwaretechnische Architektur, die bestimmte Funktionen und Klassen zur Verfügung stellt.}.
Ralph ist eine auf dem
Django-\emph{Framework}\index{Framework: Eine softwaretechnische Architektur, die bestimmte Funktionen und Klassen zur Verfügung stellt.}
basierende
\emph{DCIM}\index{DCIM: Data Center Infrastructure Management} und
\emph{CMDB}\index{CMDB: Configuration Management Database}
Softwarelösung. Haupteinsatzgebiet der Software sind vor allem
Rechenzentren mit hoher Komplexität, die externe Verwaltungsplattformen
benötigen. Zusätzlich können aber auch herkömmliche Inventardaten von
IT-spezifischen Gegenständen in die Ralph-Plattform aufgenommen werden.

Ralph wurde von der polnischen Softwarefirma ``Allegro'' entwickelt und
ist unter der Apache 2.0 Lizenz öffentlich verfügbar. Dies ermöglicht
auch Veränderungen und Erweiterungen.

Die vorliegende Diplomarbeit bietet eine Erweiterung des Ralph-Systems.

\hypertarget{begruxfcndung-der-wahl-von-django-und-ralph}{%
\subsubsection{Begründung der Wahl von Django und
Ralph}\label{begruxfcndung-der-wahl-von-django-und-ralph}}

Django bietet eine weit verbreitete Open-Source Lösung für die
Entwicklung von Web-Diensten. Bekannte Webseiten, die auf Django
basieren sind u.a. Instagram, Mozilla, Pinterest und Open Stack.
\cite{django-overview} Django zeichnet sich besonders durch die sog.
\emph{"Batteries included"}\index{Batteries included: Das standardmäßige Vorhandensein von erwünschten bzw. gängigen Features, zu Deutsch: "Batterien einbezogen"}
Mentalität aus. Das heißt, dass Django bereits die gängigsten Features
eines Webserver-Backends standardmäßig innehat. Diese sind (im Vergleich
zu beispielsweise der Alternative ``Flask'') u.a.

\begin{itemize}
\tightlist
\item
  Authentifikation und Autorisierung, sowie eine damit verbundene
  Benutzerverwaltung
\item
  Schutz vor gängigen Attacken (wie
  \emph{SQL-Injections}\index{SQL-Injections: klassischer Angriff auf ein Datenbanksystem}
  oder
  \emph{CSRF}\index{CSRF: Cross-Site-Request-Forgery - eine Angriffsart, bei dem ein Opfer dazu gebracht wird, eine von einem Angreifer gefälschte Anfrage an einen Server zu schicken.\cite{csrf}}\cite{csrf}
  , siehe \todo{[Verweis auf Kapitel Django-Forms und Templates]})
\end{itemize}

Zusätzlich bietet Ralph bereits einige Features, die die grundgegende
Führung und Verwaltung eines herkömmlichen Inventars unterstützen
(beispielsweise eine Suchfunktion mit automatischer
Textvervollständigung).

\hypertarget{kurzfassung-der-funktionsweise-von-django-und-rlaph}{%
\subsection{Kurzfassung der Funktionsweise von Django und
Rlaph}\label{kurzfassung-der-funktionsweise-von-django-und-rlaph}}

Im folgenden Kapitel wird die Funktionsweise des
Django-\emph{Frameworks}\index{Framework: Eine softwaretechnische Architektur, die bestimmte Funktionen und Klassen zur Verfügung stellt.},
sowie Ralph beschrieben. Ziel dieses Kapitels ist es, eine Wissensbasis
für die darauffolgenden Kapitel zu schaffen.

\hypertarget{datenbank-verbindung-pakete-und-tabellen-definition}{%
\subsubsection{Datenbank-Verbindung, Pakete und
Tabellen-Definition}\label{datenbank-verbindung-pakete-und-tabellen-definition}}

Folgende Datenbank-Arten werden von Django unterstützt:

\begin{itemize}
\tightlist
\item
  PostgreSQL
\item
  MariaDB
\item
  MySQL
\item
  Oracle
\item
  SQLite
\end{itemize}

Die Konfiguration der Datenbank-Verbindung geschieht unter
Standard-Django in der Datei \texttt{settings.py} , unter Ralph in der
jeweiligen Datei im Verzeichnis \texttt{settings} Eine detaillierte
Anleitung zur Verbindung mit einer Datenbank ist in der offiziellen
Django-Dokumentation\cite{django-doku-db} zu finden.

Die verschiedenen Funktionsbereiche des Servers sind in Paketen bzw.
Module gegliedert. Jedes Paket ist ein Ordner, der verschiedene Dateien
und Unterordner beinhalten kann. Die Dateinamen-Nomenklatur eines
Packets ist normalisiert\cite{django-file-nomenklatur}. Der Name eines
Pakets wird fortläufig App-Label genannt. Standardmäßig ist dieser Name
erster Bestandteil einer URL zu einer beliebigen grafischen
Administrationsoberfläche des Pakets. Pakete werden durch einen Eintrag
in die Variable \texttt{INSTALLED\_APPS} innerhalb der o.a.
Einstellungsdatei registriert. Beispiele sind die beiden durch die
vorliegende Diplomarbeit registrierten Pakete \texttt{"ralph.capentory"}
und \texttt{"ralph.stocktaking"}

Ist ein Python-Paket erfolgreich registriert, konnen in der Datei
\texttt{models.py} Datenbank-Tabellen als python Klassen (abgeleitet von
der Superklasse \texttt{Model}\cite{django-doku-models} definiert
werden. Diese Klassen werden daher fortlaufend als ``Modell''
bezeichnet. Tabellenattribute werden als Attribute dieser Klassen
definiert und sind jeweils Instanzen der Superklasse
\texttt{Field}\cite{django-doku-models}. Datenbankeinträge können
demnach als python Objekte betrachtet und behandelt werden.

Jedes Modell benötigt eine innere Klasse \texttt{Meta}. Sie beschreibt
die
\emph{Metadaten}\index{Metadaten: Daten, die einen gegebenen Datensatz beschreiben, beispielsweise der Autor eines Buches}.
Dazu gehört vor allem der von Benuzern lesbare Name des Modells
\texttt{verbose\_name}. \cite{django-doku-models-options}

\hypertarget{api-und-drf}{%
\subsubsection{API und DRF}\label{api-und-drf}}

Um Daten außerhalb der grafischen Administrationsoberfläche zu
bearbeiten, wird eine
\emph{API}\index{API: Application-Programming-Interface - Eine Schnittstelle, die die programmiertechnische Erstellung, Bearbeitung und Einholung  von Daten auf einem System ermöglicht}
benötigt. Eine besondere und weit verbreitete Form einer API ist eine
\emph{REST-API}
\index{REST-API: Representational State Transfer \emph{API}\index{API: Application-Programming-Interface - Eine Schnittstelle, die die programmiertechnische Erstellung, Bearbeitung und Einholung  von Daten auf einem System ermöglicht} - eine zustandslose Schnittstelle für den Datenaustausch zwischen Clients und Servern \cite{rest-api}}
\cite{rest-api}, die unter Django durch das integrierte \emph{DRF}
\textbackslash index\{DRF: Django REST Framework - Implementierung einer
\emph{REST-API}\index{REST-API: Representational State Transfer \emph{API}\index{API: Application-Programming-Interface - Eine Schnittstelle, die die programmiertechnische Erstellung, Bearbeitung und Einholung  von Daten auf einem System ermöglicht} unter Django}
implementiert wird.\cite{django-rest-framework} API Definitionen werden
unter Django in einem Paket in der Datei \texttt{api.py} getätigt.

\todo{Ignore below content for now}

\ldots{} irgendwo im Intro: bezüglich folgender 2 Aspekte: * Eine
Inventurfunktion für die Validierung von Gegenständen und die Verwaltung
von Änderungen die durch eine Inventur entstehen, sowie die benötigte
Kommunikation mit der Client-Applikation, die für die Durchführung von
Inventuren benutzt wird. * Eine zentralisierte Verwaltungsplatform der
in der HTL Rennweg geführten Inventarlisten, besonders der
SAP-Datenbank.
