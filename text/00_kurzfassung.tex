Aktuell ist eine Inventur an der HTL Rennweg überaus mühsam, da es dafür dreier separater Listen bedarf. Die erste Liste stammt
direkt aus dem SAP-System und beinhaltet infolgedessen Informationen über alle Gegenstände in der Organisation. Diese Liste wird fortan als 
„Primäre Liste“ bezeichnet. Das SAP-System ist eine Datenbank, 
die durch gesetzliche Vorgaben von unserer Schule verwendet werden muss. 

Bei der zweiten und dritten Liste handelt es sich um interne Listen, die sich auf die IT-bezogenen Gegenstände in der Organisation beschränken. 
Diese Listen werden fortan als „Sekundäre Liste“ bzw. „Tertiäre Liste“ bezeichnet. Logisch betrachtet handelt es sich bei der sekundären und 
tertiären Liste um eine Teilmenge der primären Liste.
Allerdings sind diese Listen nicht synchron zueinander und führen daher zahlreiche Komplikationen herbei. 
Das vorliegende Projekt soll die Schulinventarisierung erheblich erleichtern, indem es die erwähnten Listen sinnvoll vereint. 

Außerdem soll der Inventurvorgang selbst durch eine mobile Android-Applikation vereinfacht werden. Ausgedruckte Listen, die bisher dafür zum Einsatz 
kommen, sollen durch unsere Applikation ersetzt werden. Durch das Scannen von Barcodes
können Gegenstände auf der App validiert werden - mit dem angenehmen Nebeneffekt, dass der Inventurvorgang massiv beschleunigt wird. Die Barcodes werden vom SAP-System für jeden Gegenstand generiert 
und sind in Form eines Aufklebers an den Gegenständen angebracht. 
Da alle Änderungen protokolliert werden, ist ein genauer Verlauf und damit eine 
genaue Zuordenbarkeit zu im Serversystem registrierten Benutzern möglich.