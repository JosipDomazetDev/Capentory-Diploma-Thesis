The current inventory process at HTL Rennweg is extremly tedious due to the fact that three independent lists serve as 
data source for all items. The first list originates from the SAP-System and therefore, contains information about all items in our organization.
Henceforth, this list will be labeled as „Primary Source“. The SAP-System is a database that our school is required to use by law. 

The second and third lists are unofficial lists for internal usage that are limited to the IT-related items at our organization. 
Henceforth, these lists will be labeled as „Secondary Source“ and „Tertiary Source“ respectively.
The secondary and tertiary sources contain additional information. For instance, they include items that are represented by only one entry in the primary source but that are multiple different items in reality. Moreover, they specify the position of an item within a room (\eg{} a PC might be in a cabinet that itself is located in a room).
However, these sources are not in sync with each other and are thus, the cause 
of countless complications. The aim of this diploma thesis is to simplify the inventory process tremendously by unifying said sources in a reasonable manner.

Furthermore, the current inventory process itself is to be simplified by a mobile Android app. Currently used printed lists are to be replaced 
by said app. By scanning barcodes, one can validate items on the application -- with the pleasant side effect of speeding up the inventory process
considerably. Mentioned barcodes are provided for each item by the SAP-System and are physically attached to the items as stickers. 
Due to the fact that all changes are being logged, there is an exact history and accountability to users that are registered in the server system.