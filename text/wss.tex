\hypertarget{technische-umsetzung-infrastruktur}{%
\section{Technische Umsetzung:
Infrastruktur}\label{technische-umsetzung-infrastruktur}}

Um allen Kunden einen problemlosen Produktivbetrieb zu gewährleisten,
muss ein physischer Server aufgesetzt werden. Auf diesem können dann
alle Komponenten unseres Git-Repositories geklont und betriebsbereit
installiert werden. Dafür gab es folgende Punkte zu erfüllen:

\begin{itemize}
\tightlist
\item
  das Beschaffen eines Servers
\item
  das Aufsetzen eines Betriebssystemes
\item
  die Konfiguration der notwendigen Applikationen
\item
  die Konfiguration der Netzwerkschnittstellen
\item
  das Testen der Konnektivität im Netzwerk
\item
  die Einrichtung des Produktivbetriebes der Applikation
\item
  das Verfassen einer Serverdokumentation
\item
  die Absicherung der Maschine
\item
  die Überwachung des Netzwerk
\end{itemize}

\hypertarget{anschaffung-des-servers}{%
\subsection{Anschaffung des Servers}\label{anschaffung-des-servers}}

Den 5. Klassen wird, dank gesponserter Infrastruktur, im Rahmen ihrer
Diplomarbeit ein Diplomarbeitscluster zur Verfügung gestellt. Damit
können sich alle Diplomarbeitsteams problemlos Zugang zu ihrer eigenen
virtualisierten Maschine verschaffen. Die Virtualisierung dieses großen
Servercluster funktioniert mittels einer ProxMox-Umgebung.

\hypertarget{proxmox}{%
\subsubsection{ProxMox}\label{proxmox}}

Proxmox Virtual Environment (kurz PVE) ist eine auf Debian und KVM
basierende Virtualisierungs-Plattform zum Betrieb von
Gast-Betriebssystemen. Vorteile:

\begin{itemize}
\tightlist
\item
  läuft auf fast jeder x86-Hardware
\item
  frei verfügbar
\item
  ab 3 Servern Hochverfügbarkeit
\end{itemize}

Jedoch liegen alle Maschinen der Diplomarbeitsteams in einem eigens
gebauten und gesicherten Virtual Private Network (VPN), sodass nur
mittels eines eingerichteten Tools auf den virtualisierten Server
zugegriffen werden kann.

\hypertarget{forticlient}{%
\subsubsection{FortiClient}\label{forticlient}}

FortiClient ermöglicht es, VPN-Konnektivität anhand von IPsec oder SSL
zu erstellen. Die Datenübertragung wird verschlüsselt, und damit der
enstandene Datenstrom vollständig gesichert über einen sogenannten
``Tunnel'' übertragen.

Da die vorliegende Diplomarbeit die Erreichbarkeit des Servers im
Schulnetz verlangt, muss die Maschine in ausreichenden Ausmaß
abgesichert werden, damit sie ohne Bedenken in das Schulnetz gehängt
werden kann. Dafür müssen folgende Punkte gewährleistet sein:

\begin{itemize}
\tightlist
\item
  Firewallkonfiguration (\siehe{absicherung-der-virtuellen-maschine})
\item
  Wohlüberlegte Passwörter und Zugriffsrechte
\end{itemize}

\hypertarget{wahl-des-betriebssystems}{%
\subsection{Wahl des Betriebssystems}\label{wahl-des-betriebssystems}}

Neben den physischen Hardwarekomponenten wird für einen funktionierenden
und leicht bedienbaren Server logischerweise auch ein Betriebsystem
benötigt. Die erste Entscheidung, welche Art von Betriebssystem für die
Diplomarbeit in Frage kam, wurde rasch beantwortet: Linux. Während der
Schulzeit an der HTL Rennweg durfte das Diplomarbeitsteam an zwei
verschiedenen Linux-Distributionen, die auch für den Serverbetrieb der
Diplomarbeit in Frage kamen, durchführen:

\begin{itemize}
\tightlist
\item
  Linux CentOS
\item
  Linux Ubuntu
\end{itemize}

\hypertarget{linux-centos}{%
\subsubsection{Linux CentOS}\label{linux-centos}}

CentOS ist eine frei verfügbare Linux Distribution, die auf Red-Hat
\cite{redhat} aufbaut. Hinter Ubuntu und Debian ist CentOS die am
dritthäufigsten verwendete Linux-Plattform und wird von einer offenen
Gruppe von freiwilligen Entwicklern betreut, gepflegt und
weiterentwickelt.

\hypertarget{linux-ubuntu}{%
\subsubsection{Linux Ubuntu}\label{linux-ubuntu}}

Ubuntu ist die am meisten verwendete Linux-Betriebssystemsoftware für
Webserver. Auf Debian basierend, ist das Ziel der Ubuntu-Entwickler, ein
einfach zu installierendes und leicht zu bedienendes Betriebssystem mit
aufeinander abgestimmter Software zu schaffen. Hauptsponsor des
Ubuntu-Projektes ist der Software-Hersteller Canonical \cite{canonical},
der vom südafrikanischen Unternehmer Mark Shuttleworth \cite{mark}
gegründet wurde.

\hypertarget{vergleich-und-wahl}{%
\subsubsection{\texorpdfstring{Vergleich und Wahl
\cite{wahl}}{Vergleich und Wahl }}\label{vergleich-und-wahl}}

\textbf{CentOS}

\begin{itemize}
\tightlist
\item
  Kompliziertere Bedienung
\item
  Keine regelmäßigen Softwareupdates
\item
  Weniger Dokumentation vorhanden
\end{itemize}

\textbf{Ubuntu}

\begin{itemize}
\tightlist
\item
  Leichte Bedienung
\item
  Wird ständig weiterentwickelt und aktualisiert
\item
  Zahlreich brauchbare Dokumentation im Internet vorhanden
\item
  Wird speziell von Ralph empfohlen
\end{itemize}

Aus den angeführten Punkten entschied sich das Diplomarbeitsteam
klarerweise das Betriebsystem Ubuntu zu verwenden, vorallem auch weil
der Hersteller der Serversoftware ``Ralph'' die Verwendung von diesem
Betriebsystem empfiehlt. Anschließend wird die Installation des
Betriebsystemes genauer erläutert und erklärt.

\hypertarget{installation-des-betriebsystemes}{%
\subsection{Installation des
Betriebsystemes}\label{installation-des-betriebsystemes}}

Wie bereits unter Punkt ``Anschaffung des Servers''
(\siehe{anschaffung-des-servers}) erwähnt, wird uns von der Schule ein
eigener Servercluster mit virtuellen Maschinen zur Verfügung gestellt.
Durch die ProxMox-Umgebung und diversen Tools, ging die Installation der
Ubuntu-Distribution rasch von der Hand. In der Virtualisierungsumgebung
der Schule musste nur ein vorhandenes Linux-Ubuntu 18.04 ISO-File
gemountet und anschließend eine gewöhnliche Betriebsysteminstallation
für Ubuntu durchgeführt werden. Jedoch kam es beim ersten Versuch zu
Problemen mit der Konfiguration der Netzwerkschnittstellen, die im
nächsten Punkt genauer erläutert werden.

\hypertarget{internetkonnektivituxe4t-der-maschine}{%
\subsection{Internetkonnektivität der
Maschine}\label{internetkonnektivituxe4t-der-maschine}}

\hypertarget{konfiguration}{%
\subsubsection{Konfiguration}\label{konfiguration}}

Im Rahmen des Laborunterrichts an der HTL Rennweg, bekamen die Schüler
für diverse Unklarheiten ein sogenanntes Cheat-Sheet \cite{cheat} für
Linux-Befehle zur Verfügung gestellt. In diesem Cheat-Sheet finden sich
unteranderem Anleitungen für die Konfiguration einer
Netzwerkschnittstelle auf einer CentOS/RedHat sowie
Ubuntu/Debian-Distribution. Den Schülern der fünften
Netzwerktechnikklasse sollte diese Kurzkonfiguration jedoch schon leicht
von der Hand gehen, da sie diese Woche für Woche benötigen.

Eine Netzwerkkonfiguration mit statischen IPv4-Adressen für eine
Ubuntu-Distribution könnte wie folgt aussehen:

In \texttt{/etc/network/interfaces}:

\begin{verbatim}
auto ens32
iface ens32 inet static
address 192.168.0.1
netmask 255.255.255.0
\end{verbatim}

Eine Netzwerkkonfiguration mit Verwendung eines IPv4-DHCP-Servers für
eine Ubuntu-Distribution könnte wie folgt aussehen:

In \texttt{/etc/network/interfaces}:

\begin{verbatim}
auto ens32
iface ens32 inet dhcp
\end{verbatim}

\hypertarget{topologie-des-netzwerkes}{%
\subsubsection{Topologie des
Netzwerkes}\label{topologie-des-netzwerkes}}

Unter Abbildung 5.1 wird der Netzwerkplan veranschaulicht. Auf der
linken Seite wird der Servercluster der Diplomarbeitsteams dargestellt,
worauf die virtuelle Maschine der Diplomarbeit gehostet wird. In der
Mitte ist die FortiGate-Firewall zu sehen, die nicht nur als äußerster
Schutz vor Angriffen dient, sondern auch die konfigurierte
VPN-Verbindung beinhaltet und nur berechtigten Teammitgliedern den
Zugriff gewährleistet. Desweiteren ist die moderne Firewall auch für die
Konnektivität der virtuellen Maschine im Schulnetz zuständig, aber dazu
später (unter Punkt ``Absicherung der virtuellen Maschine'') mehr.

\begin{figure}[ht]
\centering
\includegraphics{topo1.png}
\caption{Netzwerkplan}
\end{figure}

\hypertarget{installation-der-notwendigen-applikationen}{%
\subsection{Installation der notwendigen
Applikationen}\label{installation-der-notwendigen-applikationen}}

Damit die Ubuntu-Maschine für den Produktivbetrieb startbereit ist,
müssen im Vorhinein noch einige wichtige Konfigurationen durchgeführt
werden. Die wichtigste (Netzwerkkonfiguration) wurde soeben ausführlich
erläutert, doch ohne der Installation von diversen Applikationen, wäre
das System nicht brauchbar.

\hypertarget{advanced-packaging-tool}{%
\subsubsection{Advanced Packaging Tool}\label{advanced-packaging-tool}}

Mit diesem Tool werden auf dem System die notwendigen Applikationen
heruntergeladen, extrahiert und anschließend installiert. Insgesamt
stehen einem 18 apt-get commands zur Verfügung. Genauere Erklärungen zu
den wichtigsten commands folgen.

\hypertarget{apt-get-update}{%
\paragraph{apt-get update}\label{apt-get-update}}

Update liest alle in \texttt{/etc/apt/sources.list}, sowie in
\texttt{/etc/apt/sources.list.d/} eingetragenen Paketquellen neu ein.
Dieser Schritt wird vor allem vor einem upgrade-command, oder nach dem
Hinzufügen einer neuen Quelle empfohlen, um sich die neusten
Informationen für Pakete ansehen zu können.

\hypertarget{apt-get-upgrade}{%
\paragraph{apt-get upgrade}\label{apt-get-upgrade}}

Upgrade bringt alle bereits installierten Pakete auf den neuesten Stand.

\hypertarget{apt-get-install}{%
\paragraph{apt-get install}\label{apt-get-install}}

Install lädt das Paket bzw. die Pakete inklusive der noch nicht
installierten Abhängigkeiten (und eventuell der vorgeschlagenen weiteren
Pakete) herunter und installiert diese. Außerdem besteht die
Möglichkeit, beliebig viele Pakete auf einmal anzugeben, indem sie
mittels eines Leerzeichens getrennt werden.

\hypertarget{apt-get-remove}{%
\paragraph{apt-get remove}\label{apt-get-remove}}

Mit apt-get remove werden nicht mehr benötigte Pakete von einem
Ubuntu-System vollständig entfernt.

\hypertarget{installierte-pakete}{%
\subsubsection{Installierte Pakete}\label{installierte-pakete}}

\hypertarget{nginx}{%
\paragraph{NGINX}\label{nginx}}

NGINX ist einer der am Häufigsten verwendeten OpenSource-Webserver unter
Linux für diverse Webanwendungen. Große Unternehmen wie Cisco,
Microsoft, Facebook oder auch IBM verwenden diesen Webserver für deren
Zwecke. Unteranderem wird NGINX auch als Reverse-Proxy, HTTP-Cache und
Load-Balancer verwendet. Wie genau NGINX für den Produktivbetrieb
funktioniert wird in einem eigenen Punkt
(\siehe{funktionsweise-der-produktivumgebung}) erläutert.

\hypertarget{docker}{%
\paragraph{Docker}\label{docker}}

Docker ist eine frei verwendbare Software, die die Erstellung und den
Betrieb von Linux Containern ermöglicht. Wie genau dies funktioniert,
wird später etwas später (\siehe{funktionsweise-der-produktivumgebung})
genauer beschrieben und erklärt.

\hypertarget{docker-compose}{%
\paragraph{docker-compose}\label{docker-compose}}

Die Verwaltung und Verlinkung von mehreren Containern kann auf Dauer
sehr nervenaufreibend sein. Die Lösung dieses Problems nennt sich
docker-compose. Wie docker-compose jedoch genau funktioniert, wird
ebenfalls wie das Grundkonzept von Docker
(\siehe{funktionsweise-der-produktivumgebung}) präziser erläutert.

\hypertarget{mysql}{%
\paragraph{MySQL}\label{mysql}}

MySQL ist ein OpenSource-Datenverwaltungssystem und die Grundlage für
die meisten dynamischen Websiten. Darauf werden die Inventurdatensätze
der HTL Rennweg gespeichert. Nähere Informationen finden sich ebenfalls
während der Erklärung der Funktionsweise des Produktivbetriebes
(\siehe{funktionsweise-der-produktivumgebung}) wieder.

\hypertarget{redis}{%
\paragraph{Redis}\label{redis}}

Redis ist eine In-Memory-Datenbank mit einer
Schlüssel-Wert-Datenstruktur (Key Value Store). Wie auch MySQL handelt
es sich um eine OpenSource-Datenbank.

\hypertarget{nagios}{%
\paragraph{Nagios}\label{nagios}}

Nagios ist ein Monitoring-System, mit dem sich verschiedene Geräte und
auf solche laufende Dienste (oder auch Eigenschaften) überwachen lassen.
Ziel ist es dabei schnell Ausfälle festzustellen und diese dem
zuständigen Administrator mit zu teilen, so dass dieser dann schnell
darauf reagieren kann.

\hypertarget{virtualenv}{%
\paragraph{virtualenv}\label{virtualenv}}

Bei virtualenv handelt es sich um ein Tool, mit dem eine isolierte
Python-Umgebung erstellt werden kann. Eine solch isolierte Umgebung
besitzt eine eigene Installation von diversen Services und teilt ihre
libraries nicht mit anderen virtuellen Umgebungen (im optionalen Fall
greifen sie auch nicht auf die global installierten libraries zu). Dies
bringt vor allem den großen Vorteil, dass im Testfall virtuelle
Umgebungen aufgesetzt werden können, um nicht die globalen
Konfigurationen zu gefährden.

\hypertarget{python}{%
\paragraph{Python}\label{python}}

Python ist einer der Hauptbestandteile auf dem Serversystem der
Diplomarbeit. Das Backend (=Serveranwendung) basiert wie bereits erwähnt
auf dem Python-Framework ``Django''. Um dieses Framework auf dem System
installieren zu können wird jedoch noch ein weiteres ``Packaging-Tool'',
speziell für Python-Module, benötigt.

\hypertarget{pip}{%
\paragraph{pip}\label{pip}}

Und dieses Tool nennt sich Pip. Pip ist ein rekursives Akronym für
\textbf{P}ip \textbf{I}nstalls \textbf{P}ython und ist, wie bereits
erwähnt, das Standardverwaltungswerkzeug für Python-Module. Die Funktion
sowie Syntax kann relativ gut mit der von apt verglichen werden.

\hypertarget{uwsgi}{%
\paragraph{uWSGI}\label{uwsgi}}

Das eigentliche Paket, mit dem der Produktivbetrieb schlussendlich
gewährleistet wurde, nennt sich uWSGI. Speziell wurde es für die
Produktivbereitstellung von Serveranwendungen (wie eben der
Django-Server der vorliegenden Diplomarbeit) entwickelt und harmoniert
eindrucksvoll mit der Webserver-Software NGINX. Die grundlegende
Funktionsweise von uWSGI, sowie eine Erklärung, warum schlussendlich
diese Software und nicht Docker verwendet wurde, wird in einem eigenen
Teil (\siehe{probleme-der-produktivumgebung}) veranschaulicht.

\hypertarget{django}{%
\paragraph{Django}\label{django}}

Django ist ein in Python geschriebenes Webframework, auf dem unsere
Serveranwendung basiert. Genauere Informationen wurden jedoch schon
übermittelt.
(\siehe{kurzfassung-der-funktionsweise-von-django-und-ralph})

\hypertarget{runsslserver}{%
\paragraph{runsslserver}\label{runsslserver}}

Runsslserver ist ein Python-Paket mit dem eine Entwicklungsumgebung über
https erreichbar gemacht werden kann.

\hypertarget{produktivbetrieb-der-applikation}{%
\subsection{Produktivbetrieb der
Applikation}\label{produktivbetrieb-der-applikation}}

Das Aufsetzen beziehungsweise die Installation der Produktivumgebung ist
der wichtigste, aber auch aufwendigste, Bestandteil jeder
Serverinfrastruktur. Eine Produktivumgebung soll von einer Testumgebung
möglichst weit getrennt sein, damit die zu testende Software keinen
Schaden für den produktiven Betrieb anrichten kann. Weiters soll durch
den Produktivbetrieb der Server um einiges perfomanter sein, da dieser,
im Falle der vorliegenden Diplomarbeit, einen optimierten Webserver
(NGINX) verwendet.

\hypertarget{entwicklungsumgebung}{%
\subsubsection{Entwicklungsumgebung}\label{entwicklungsumgebung}}

Die Entwickler des Grundservers ``Ralph'' haben eine eigene
Entwicklungsumgebung für den Django-Server erstellt. Normalerweise
findet sich in einem klassischen Python-Projekt oder einem Projekt, dass
auf einem Python-Framework (wie zum Beispiel Django) basiert, ein
sogenanntes ``\texttt{manage.py}''-File. Hierbei handelt es sich um ein
Script, dass das Management eines beliebigen Python-Projektes
unterstützt. Mit dem Script ist man unteranderem in der Lage, den
Webserver auf einem unspezifischen Rechner zu starten, ohne etwas
Weiteres installieren zu müssen.

Jedoch wurde diese Datei im Vorgängerprojekt in eine eigene spezifische
Entwicklungsumgebung umimplementiert. Der Server startet sich nach der
eigenen Installation (welche ab Seite 4 im Dokument "Serverdokumentation
Schritt für Schritt erklärt wird) nicht mehr mittels:

\begin{verbatim}
python manage.py runserver
\end{verbatim}

sondern mit:

\begin{verbatim}
dev_ralph runserver 0.0.0.0:8000
\end{verbatim}

Obwohl die Befehle dieses Serverstarts nicht wirklich ident aussehen,
führen sie im Hintergrund aber die gleichen Unterbefehle aus, um eine
sichere Verwendung der Entwicklungsumgebung zu gewährleisten.

Da das Diplomarbeitsteam diese Entwicklungsumgebung vorerst auf dem
Produktivserver für Testzwecke verwendete, wurde ein eher unbekanntes
Paket namens ``runsslserver'' für die Entwicklungsumgebung installiert
und eingebaut. Mit diesem Tool kann eine Django-Entwicklungsumgebung
mittels ``https'' gestartet werden. Zuerst musste, üblich um eine
https-Verbindung einzurichten, beispielsweise mit dem Befehl

\begin{verbatim}
sudo openssl req -x509 -nodes -days 365 -newkey rsa:2048 
-keyout /ssl/nginx.key -out /ssl/cert.crt
\end{verbatim}

ein Zertifikat mit dem zugehörigen Schlüssel generiert werden.

Kurzerklärung der Befehlsoptionen

\begin{itemize}
\tightlist
\item
  \texttt{-x509} Gibt an, statt eines CSR gleich ein selbstsigniertes
  Zertifikat auszustellen.
\item
  \texttt{-nodes} Zertifikat wird nicht über ein Kennwort geschützt.
  Damit kann der Server ohne weitere Aktion (Eingabe des Kennworts)
  gestartet werden.
\item
  \texttt{-days} Beschreibt die Gültigkeit des Zertifikates für 365
  Tage.
\item
  \texttt{-newkey\ rsa:2048} Generiert das Zertifikat und einen 2048-bit
  langen RSA Schlüssel.
\item
  \texttt{-keyout} Gibt die Ausgabepfad und -datei für den Schlüssel an.
\item
  \texttt{-out} Gibt die Ausgabepfad und -datei des Zertifikates an.
\end{itemize}

Der Befehl ``runsslserver'' muss jetzt nur noch in den
``Installed-Apps'' des Projektes (im vorliegenden Projekt befinden sich
diese in der Datei ``\texttt{base.py}'') hinzugefügt werden.

Dann kann der Server problemlos mit

\begin{verbatim}
dev_ralph runsslserver 0.0.0.0:443
\end{verbatim}

gestartet werden.

\hypertarget{probleme-der-produktivumgebung}{%
\subsubsection{Probleme der
Produktivumgebung}\label{probleme-der-produktivumgebung}}

Zu Beginn dachte das Diplomarbeitsteam, dass die Umsetzung des
Django-Servers in eine Produktivumgebung nicht allzu kompliziert wäre,
da von Ralph bereits Dockerfiles für den Produktivbetrieb vorlagen.
Dadurch wurde zu Beginn der Arbeit versucht dieses bereits existierendes
Dockersystem zu starten, jedoch war im Browser dann nicht der
Diplomarbeitsserver, sondern die Basislösung des Vorgängers zu sehen.
Der Infrastrukturverantwortliche analysierte anschließend die
Docker-Dateien und musste feststellen, dass sie für die Zwecke der
Diplomarbeit so tatsächlich nicht verwendet werden können. Ralph hat in
Verbindung mit dem implementierten Dockersystem einige komplexe,
aufeinander zugreifende Skripts verwendet. Vorweg muss gesagt werden,
dass der Vorgänger deren Serverlösung als Service anbietet und in den
vorliegenden Skripts deren Server ohne den Änderungen des
Diplomarbeitsteam installiert. Daraufhin wird aus dem Webservice ein
Docker-Container erstellt und anschließend mit den anderen
Docker-Komponenten hochgefahren. Dadurch war im Webbrowser beim
Serverabruf, statt der gewollten Serverlösung, die falsche zu sehen. Der
Fortschritt des Arbeitspaketes der Produktivumgebung war somit wieder um
einiges geschrumpft und es musste schnellstmöglich eine Alternative
gefunden werden, da das Umschreiben der Dockerfiles, beziehungsweise der
Skripts, ein riesiger Aufwand wäre.

Nach kurzer Recherche stieß man schließlich auf zwei verwendbare
Alternativen, nachfolgend genannt und verglichen werden.

\hypertarget{alternativen}{%
\subsubsection{Alternativen}\label{alternativen}}

Es wurde bereits erläutert, warum das Diplomarbeitsteam die Dockerlösung
des Vorgängers nicht verwendet. (\siehe{probleme-der-produktivumgebung})
Allerdings musste rasch eine Alternative für die Umsetzung des
Produktivbetriebs gesucht werden, um möglichst wenig Zeit zu verlieren.
Zwei Alternativen, die für den Produktivbetrieb in Frage kamen, wurden
genauer analysiert: uWSGI sowie Gunicorn.

\hypertarget{uwsgi-vs.-gunicorn}{%
\subsubsection{uWSGI vs.~Gunicorn}\label{uwsgi-vs.-gunicorn}}

Bei beiden Alternativen handelt es sich um
Schnittstellen-Spezifikationen, unteranderem für die Programmiersprache
Python, die eine Schnittstelle zwischen Webservern und Webframeworks
bzw. Web Application Servern festlegen, um die Portabilität von
Webanwendungen auf unterschiedlichen Webservern zu fördern.

Gunicorn ist ein Pre-Fork-Worker-Modell\cite{prefork}, das aus Rubys
Unicorn-Projekt portiert wurde. Der Gunicorn-Server ist weitgehend
kompatibel mit verschiedenen Web-Frameworks, einfach implementiert,
spart Serverressourcen und ist recht schnell.

Das uWSGI-Projekt zielt darauf ab, einen vollständigen Stack für den
Aufbau von Hosting-Diensten zu entwickeln.

\hypertarget{wahl}{%
\subsubsection{Wahl}\label{wahl}}

Da beide Alternativen sehr ähnlich sind, lag es am
Infrastrukturverantwortlichen, welche für die Installation der
Produktivumgebung ausgewählt wird. Zuerst wurde versucht, alles mithilfe
von Gunicorn aufzusetzen. Da dies jedoch mehrmalig Probleme verursachte,
entschied man sich für die Verwendung von uWSGI. Nach kurzer Recherche
stieß der Infrastrukturverantwortliche auf ein vielversprechendes
Tutorial im Internet\cite{tutorialuwsgi}, mit dem die Installation
reibungslos verlief. Die genaue Installationsanleitung wurde bereits in
der Serverdokumentation niedergeschrieben. Die Funktionsweise, also wie
uWSGI genau arbeitet, wirdnachfolgend (\siehe{funktion-von-uwsgi})
anhand einer Grafik erklärt.

\hypertarget{funktionsweise-der-produktivumgebung}{%
\subsubsection{Funktionsweise der
Produktivumgebung}\label{funktionsweise-der-produktivumgebung}}

\hypertarget{funktion-von-uwsgi}{%
\paragraph{Funktion von uWSGI}\label{funktion-von-uwsgi}}

Es wurde bereits desöfteren erklärt, warum die Dockerlösung des
Vorgängers vom Diplomarbeitsteam nicht verwendet wird.
(\siehe{probleme-der-produktivumgebung}) Jedenfalls wurde nun die
Einrichtung der Produktivumgebung mittels uWSGI erfolgreich
durchgeführt. In der folgenden Grafik wird die Funktion von uWSGI
genauer dargestellt.

\begin{figure}[ht]
\centering
\includegraphics{uwsgi.png}
\caption{Funktionsweise von uWSGI}
\cite{uwsgi}
\end{figure}

\begin{enumerate}
\def\labelenumi{\arabic{enumi}.}
\tightlist
\item
  Ein beliebiger Benutzer eines Webbrowsers (zum Beispiel Google Chrome
  oder Mozilla Firefox) sendet einen sogenannten ``Webrequest'' auf den
  https-Port ``443'' (\siehe{probleme-der-produktivumgebung}).
\item
  Ein beliebig gewählter, optimierter Webserver (im Fall der
  vorliegenden Diplomarbeit ``NGINX'') stellt Dateien wie Javascript,
  CSS oder auch Bilder bereit und macht diese für den Benutzer abrufbar.
\item
  Hier wird die Kommunikation zwischen dem hochperfomanten Webserver
  NGINX und dem Webinterface uWSGI mit Verwendung eines klassischen
  Websockets veranschaulicht. (Anm.: Bei einem Websocket handelt es sich
  um ein auf TCP basierendes Protokoll, dass eine bidirektionale
  Verbindung zwischen einer Webanwendung und einem Webserver herstellt).
\item
  Das eigentliche Interface von uWSGi ist hier zu sehen. Diese
  Schnittstelle sorgt für die Kommunikation des oben genannten
  Websockets mit dem verwendeten Python-Framework.
\item
  Django reagiert nun auf die Anfrage des Benutzers und lässt diesem
  (falls dessen Zugriffsrechte darauf es erlauben) die gewünschten
  Daten.
\item
  Die vom Benutzer gewünschten Daten werden in einer MySQL-Datenbank
  gespeichert.
\end{enumerate}

Grundsätzlich kann die Grafik auch mittels

\begin{verbatim}
the web client <-> the web server <-> the socket <-> uwsgi <-> Django
\end{verbatim}

als normale ASCII-Zeichenkette dargestellt werden.

\hypertarget{funktion-von-docker}{%
\paragraph{Funktion von Docker}\label{funktion-von-docker}}

Docker wird im Projekt ausschließlich für die Bereitstellung der
Datenbank verwendet. Der Befehl

\texttt{docker-compose\ -f\ docker/docker-compose-dev.yml\ up\ -d}

im Projektstammverzeichnis erstellt und startet die in der folgenenden
Grafik zu sehenden Container, die das Datenbanksystem darstellen:

\begin{figure}[ht]
\centering
\includegraphics{docker.png}
\caption{Datenbanksystem mit Docker}
\end{figure}

Die erstellten Container enthalten die Datenbankanwendungen, sowie deren
Ressourcen und speichern, beziehungsweise stellen eine dauerhafte
Verfügbarkeit der erfassten Inventurdaten bereit.

\hypertarget{erreichbarkeit-mittels-https}{%
\subsubsection{Erreichbarkeit mittels
HTTPS}\label{erreichbarkeit-mittels-https}}

Weiters wurde NGINX wurde mit einem sogenannten Self Signed Certificate
ausgestattet, um eine sichere Verbindung des Benutzers mit dem Webserver
zu gewährleisten. Die Konfiguration von NGINX für den Gebrauch von uWSGI
mit HTTPS ist auf Seite 9 der Serverdokumentation zu finden. Im
Webbrowser ``Microsoft Edge'' würde der Aufruf des Servers nun wie folgt
aussehen:

\begin{figure}[ht]
\centering
\includegraphics{https.png}
\caption{Aufruf des Servers über HTTPS}
\end{figure}

Der Zertifikatsfehler, der am Bild deutlich zu sehen ist, bedeudet
jedoch nichts anderes als, dass das Zertifakt des Produktivservers der
Diplomarbeit nicht von einer offiziellen Zertifizierungsstelle signiert
wurde. Da der Server aber sowieso nur im Schulnetz der HTL 3 Rennweg
beziehungsweise über einen VPN-Tunnel erreichbar ist, war es nicht
nötig, sich an solch eine Zertifizierungsstelle zu wenden.

\hypertarget{verwendung-einer-.ini-datei}{%
\subsubsection{Verwendung einer
.ini-Datei}\label{verwendung-einer-.ini-datei}}

Bei einem File mit einer .ini-Endung handelt es sich um eine
Initialisierungsdatei. Diese beinhaltet beispielsweise
Konfigurationsmöglichen die zum Start eines bestimmten Dienstes benötigt
werden. Glücklicherweise funktioniert so ein File auch wunderbar mit
uWSGI. Durch die Implementierung einer ralph.ini-Datei wurden die
Befehlsoptionen des \texttt{uwsgi}-Befehls ausgelagert und dieser somit
für den Serverstart verkürzt.

Ein kleiner Vergleich:

\begin{verbatim}
uwsgi --socket mysite.sock --module mysite.wsgi --chmod-socket=664 --proceses 10

uwsgi --ini ralph.ini
\end{verbatim}

Beide Befehle liefern schlussendlich das gleiche Ergebnis, jedoch ist
der unter Befehl (Verwendung einer .ini-Datei) um einiges kürzer und
erspart somit nervernaufreibende Schreibarbeit.

\hypertarget{neustartverhalten-mittels-service}{%
\subsubsection{Neustartverhalten mittels
Service}\label{neustartverhalten-mittels-service}}

Für den Gebrauch von uWSGi gibt es einige Lösungen, um den automatischen
Neustart, beispielsweise bei Eintritt eines Stromausfalls, zu
gewährleisten. Das Team hat sich, mit der Implementierung eines eigenen
``Ralph-Service'', für den klassischen Linuxweg entschieden.

\hypertarget{systemd}{%
\paragraph{Systemd}\label{systemd}}

System gibt es seit 2010 und wurde ursprünglich von Lennart Poettering
(Red Hat Inc.) entwickelt. Es ist heutzutage der de-facto Standard,
daher in nahezu allen Distros standardmäßig inkludiert und hat damit
System-V-Init abgelöst (ist aber noch zu 99\% mit System-V-Init Skripten
abwärtskompatibel). Die ursprüngliche Überlegung: ein schnellerer
Systemstart durch Parallelisierung und Verwenden von kompiliertem C-Code
statt Shell-Boot-Skripten. Mittlerweile ist Systemd nicht NUR Service-,
sondern ganzer System-Manager und daher viel mächtiger als Init-V.
Systemd wird als erster Prozess vom Linux-Kernel gestartet und hat daher
die PID 1.

\hypertarget{service-datei}{%
\paragraph{.service-Datei}\label{service-datei}}

\begin{verbatim}
[Unit]
Description=ralphserver
After=network.target

[Service]
ExecStart=/usr/bin/uwsgi /home/capentory/capentory-prod/
          capentory-ralph/ralph/ralph.ini

Restart=on-failure
RestartSec=60s

[Install]
WantedBy=multi-user.target
\end{verbatim}

\begin{itemize}
\tightlist
\item
  \textbf{Description}: Kurzbeschreibung des Dienstes
\item
  \textbf{After=network.target}: Der Dienst wird gestartet, sobald eine
  Netzwerkverbindung besteht
\item
  \textbf{ExecStart}: Den auszuführenden Shell-Befehl (in diesem Fall
  wird der uWSGI-Server gestartet)
\item
  \textbf{Restart und RestartSec}: Hier wird der Dienst beispielsweise
  bei einem unsauberen Beenden des Prozesses, einem Timeout o.ä.
  neugestartet. Es ist auch möglich, einen Dienst neu zu starten, wenn
  der Watchdog einen Fehler meldet.
\item
  \textbf{WantedBy=multi-user.target}: Normalbetrieb
\end{itemize}

\hypertarget{start-des-service}{%
\paragraph{Start des Service}\label{start-des-service}}

Nach dem korrekten Erstellen der \texttt{ralph.service}-Datei kann
dieser als ganz normaler Linux-Service behandelt werden. Mit

\begin{verbatim}
systemctl enable ralph
systemctl start ralph
\end{verbatim}

wird der angelegte Service nun immer, wenn die virtuelle Maschine
gestartet wurde, ausgeführt.

\hypertarget{absicherung-der-virtuellen-maschine}{%
\subsection{Absicherung der virtuellen
Maschine}\label{absicherung-der-virtuellen-maschine}}

Ein weiterer wichtiger und sensibler Teil der Serverinfrastruktur ist
die Absicherung der virtuellen Maschine gegen Angriffe. Da es sich im
Rahmen der Diplomarbeit um geheime Daten der Schulinventur handelt, war
es ein Anliegen der Verantwortlichen, dass mit diesen Daten
verantwortungsvoll und vorsichtig umgegangen wird.

\hypertarget{firewall}{%
\subsubsection{Firewall}\label{firewall}}

Weiter oben (\siehe{topologie-des-netzwerkes}) wird der Plan des
Netzwerkes veranschaulicht. Die in der Mitte liegende FortiGate-Firewall
stellt, wie bereits in oben genannten Punkt erwähnt, die Verfügbarkeit
des Servers im Schulnetz bereit. Dadurch dass der Server nur über eine
VPN-Verbindung konfiguriert werden kann, denkt man sich bestimmt, dass
die virtuelle Maschine schon genug gesichert sei. Jedoch ist es sinnvoll
den Server doppelt abzusichern und somit wurden auf der Linux-Maschine
ebenfalls noch Firewallregeln konfiguriert.

\hypertarget{erlauben-einer-ssh-verbindung}{%
\paragraph{Erlauben einer
SSH-Verbindung}\label{erlauben-einer-ssh-verbindung}}

Die virtuelle Maschine wurde aufgrund der nicht-vorteilshaften Konsole
von ProxMox immer über SSH mit einer beliebigen externen Konsole
(beispielsweise Putty) konfiguriert. Dies ist wegen der
FortiGate-Konfiguration nur mit VPN-Verbindung möglich. Direkt auf dem
Server wurde daher eine Firewallregel für die Erlaubnis von SSH
implementiert.

Regel: \texttt{sudo\ ufw\ allow\ ssh}

oder: \texttt{sudo\ ufw\ allow\ 22}

\hypertarget{erlauben-des-datenaustausches-mittels-http}{%
\paragraph{Erlauben des Datenaustausches mittels
HTTP}\label{erlauben-des-datenaustausches-mittels-http}}

Obwohl der Webserver den Datenaustausch mit HTTPS bevorzugt, wurde auf
der zweiten Ebene auch eine Regel für die HTTP-Verbindung konfiguriert.

Regel: \texttt{sudo\ ufw\ allow\ http}

oder: \texttt{sudo\ ufw\ allow\ 80}

\hypertarget{erlauben-des-datenaustausches-mittels-https}{%
\paragraph{Erlauben des Datenaustausches mittels
HTTPS}\label{erlauben-des-datenaustausches-mittels-https}}

Für den Datenaustausch über HTTPS musste ebenso eine Regel aktiviert
werden.

Regel: \texttt{sudo\ ufw\ allow\ https}

oder: \texttt{sudo\ ufw\ allow\ 443}

\hypertarget{verbieten-aller-restlichen-verbindungen}{%
\paragraph{Verbieten aller restlichen
Verbindungen}\label{verbieten-aller-restlichen-verbindungen}}

Zuguterletzt müssen alle restlichen (die nicht von Administratoren
gebrauchten) Verbindungen deaktiviert werden um Angriffslücken zu
schließen.

Regel: \texttt{ufw\ default\ deny}

\hypertarget{muxf6gliche-angriffsszenarien}{%
\subsubsection{Mögliche
Angriffsszenarien}\label{muxf6gliche-angriffsszenarien}}

Da der Server im Schulnetz erreichbar ist, gelten unteranderem auch die
Schüler der HTL 3 Rennweg als potenzielle Angreifer.

\hypertarget{malware}{%
\paragraph{Malware}\label{malware}}

Bei Malware handelt es sich um Schadsoftware zudem unteranderem
\textbf{Viren}, \textbf{Würmer} und \textbf{Trojaner} zählen.

\hypertarget{angriffe-auf-passwuxf6rter}{%
\paragraph{Angriffe auf Passwörter}\label{angriffe-auf-passwuxf6rter}}

Neben dem Raten und Ausspionieren von Passwörtern ist die Brute Force
Attacke weit verbreitet. Bei dieser Attacke versuchen Hacker mithilfe
einer Software, die in einer schnellen Abfolge verschiedene
Zeichenkombinationen ausprobiert, das Passwort zu knacken. Je einfacher
das Passwort gewählt ist, umso schneller kann dieses geknackt werden.

Vorbeugung des Diplomarbeitsteams: Die festgelegten Passwörter sind
komplex aufgebaut und sicher in den Köpfen der Mitarbeiter gespeichert.

\hypertarget{man-in-the-middle-attacken}{%
\paragraph{Man-in-the-middle
Attacken}\label{man-in-the-middle-attacken}}

Bei der „Man in the Middle``-Attacke nistet sich ein Angreifer zwischen
den miteinander kommunizierenden Rechnern. Diese Position ermöglicht
ihm, den ausgetauschten Datenverkehr zu kontrollieren und zu
manipulieren. Er kann \zB die ausgetauschten Informationen abfangen,
lesen, die Weiterleitung kappen usw. Von all dem erfährt der Empfänger
aber nichts.

Vorbeugung des Diplomarbeitsteams: Der Datenaustausch verläuft über
HTTPS und somit verschlüsselt.

\hypertarget{sniffing}{%
\paragraph{Sniffing}\label{sniffing}}

Unter Sniffing (Schnüffeln) wird das unberechtigte Abhören des
Datenverkehrs verstanden. Dabei werden oft Passwörter, die nicht oder
nur sehr schwach verschlüsselt sind, abgefangen. Andere Angreife
bedienen sich dieser Methode um rausfinden zu können, welche Teilnehmer
über welche Protokolle miteinander kommunizieren. Mit den so erlangten
Informationen können die Angreifer dann den eigentlichen Angriff
starten.

Vorbeugung des Diplomarbeitsteams: Der Datenaustausch verläuft über
HTTPS und somit verschlüsselt.

\hypertarget{uxfcberwachung-des-netzwerks}{%
\subsection{Überwachung des
Netzwerks}\label{uxfcberwachung-des-netzwerks}}

Sollte der Fall eintreten, dass der Server nicht mehr erreichbar ist,
wurde ein eigener Monitoring-Server im Netzwerk eingehängt und
installiert. Dadurch wird der Produktivserver rund um die Uhr überwacht.
Weiters wird das Diplomarbeitsteam bei etwaigen Komplikationen per
E-Mail benachrichtigt, damit das aufgetretene Problem beziehungsweise
die aufgetretenen Probleme möglichst schnell behoben werden können.

\hypertarget{topologieuxe4nderung}{%
\subsubsection{Topologieänderung}\label{topologieuxe4nderung}}

Im Netzwerk wurde ein zweiter Server installiert, der mit dem
OpenSource-Programm ``Nagios'' als Monitoring-Maschine für den
Produktivserver dient.

\begin{figure}[ht]
\centering
\includegraphics{neutopo.png}
\caption{Ergänzter Netzwerkplan}
\end{figure}

\hypertarget{nagios-1}{%
\subsubsection{Nagios}\label{nagios-1}}

Worum es sich bei diesem tollen Tool handelt wurde bereits bei den
installierten Paketen (\siehe{nagios}) erklärt. Nun folgt ein
vertiefender Einblick in diese Monitoring-Software.

\hypertarget{hosts}{%
\paragraph{Hosts}\label{hosts}}

Hosts sind bei Nagios definierte virtuelle Maschinen, die überwacht
werden sollen. Die vorliegende Diplomarbeit überwacht nicht nur den
Produktivserver, sondern auch den aufgesetzten Nagios-Server selbst.
Falls dieser Probleme aufweist, wird das Team ebenfalls per E-Mail
benachrichtigt aber dies wird in einem eigenen Punkt
(\siehe{notifications}) näher erläutert.

Im Webbrowser sehen die zu überwachenden Hosts wie folgt aus:

\begin{figure}[ht]
\centering
\includegraphics{hosts.png}
\caption{Die definierten Hosts der Diplomarbeit}
\end{figure}

\textbf{Capentory} ist der Name des Hosts des Produktivservers.

\textbf{Localhost} heißt der Host des Nagios-Servers.

Momentan scheint alles ohne Probleme zu funktionieren. Jedoch kann sich
soetwas schlagartig ändern:

\begin{figure}[ht]
\centering
\includegraphics{dedprod.png}
\caption{Der heruntergefahrene Produktivserver}
\end{figure}

Hier wird deutlich gemacht, dass der Produktivserver abgestürzt ist.

\hypertarget{services}{%
\paragraph{Services}\label{services}}

Nagios-Services sind, wie der Name schon verrät, die einzelnen zu
überwachenden Services eines Hosts. Zurzeit wird am Nagios-Server aus
Testzwecken eine Menge an Services überwacht. Auf dem Produktivserver
hingegen werden nur Probleme, die mit der Verbindung über Port 22 (SSH)
oder Port 443 (HTTPS) zu tun haben, erfasst.

So sehen die überwachten Services am Admin-Dashboard des Nagios-Servers
aus:

\begin{figure}[ht]
\centering
\includegraphics{services.png}
\caption{Die überwachten Services}
\end{figure}

Im Moment weist kein Service der beiden Hosts ein kritisches Problem
auf. Auf dem Nagios-Server ist der HTTP-Service zwar gelb markiert,
hierbei handelt es sich jedoch nur um eine harmlose Warnung.

Wenn auf dem Produktivserver hingegen der Webserver NGINX ausfällt sieht
das Dashboard jedoch wiederum anders aus:

\begin{figure}[ht]
\centering
\includegraphics{dednginx.png}
\caption{Die überwachten Services}
\end{figure}

\hypertarget{notifications}{%
\paragraph{Notifications}\label{notifications}}

Zuguterletzt gibt es noch das Feature der Notifications. Falls ein Host
oder ein Service der Hosts ausfällt, benachrichtigt der Nagios-Server
das Diplomarbeitsteam per E-Mail über die aufgetretenen Probleme.

\hypertarget{verfassen-einer-serverdokumentation}{%
\subsection{Verfassen einer
Serverdokumentation}\label{verfassen-einer-serverdokumentation}}

Die Funktionsweise (\siehe{produktivbetrieb-der-applikation}) wurde
bereits erklärt, allerdings war das letzte Ziel der Infrastruktur, den
Installationsvorgang der Entwicklungs- sowie der Produktivumgebung zu
dokumentieren. Somit kann nun jeder Interessierte die vorliegende
Inventurlösung verwenden und anhand der Serverdokumentation
Schritt-für-Schritt selbständig aufsetzen.
