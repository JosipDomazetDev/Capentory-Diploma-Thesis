\hypertarget{technische-umsetzung-infrastruktur}{%
\section{Technische Umsetzung:
Infrastruktur}\label{technische-umsetzung-infrastruktur}}

Um allen Kunden einen problemlosen Produktivbetrieb zu gewährleisten
muss ein physischer Server aufgesetzt werden. Auf diesem können dann
alle Komponenten unseres Git-Repositories geklont und betriebsbereit
installiert werden. Dafür gab es folgende Punkte zu erfüllen:

\begin{itemize}
\tightlist
\item
  das Organisieren eines Servers
\item
  das Aufsetzen eines Betriebssystemes
\item
  die Konfiguration der notwendigen Applikationen
\item
  die Konfiguration der Netzwerkschnittstellen
\item
  das Testen der Konnektivität im Netzwerk
\item
  die Einrichtung des Produktivbetriebes der Applikation
\item
  das Verfassen einer Serverdokumentation
\item
  die Absicherung der Maschine
\item
  die Überwachung des Netzwerks
\end{itemize}

In den folgenden Kapitel wird deutlich gemacht, wie die oben angeführten
Punkte, im Rahmen der Diplomarbeit ``Capentory'' abgearbeitet wurden.

\hypertarget{anschaffung-des-servers}{%
\subsection{Anschaffung des Servers}\label{anschaffung-des-servers}}

Den 5. Klassen wird, dank gesponserter Infrastruktur, im Rahmen ihrer
Diplomarbeit ein Diplomarbeitscluster zur Verfügung gestellt. Damit
können sich alle Diplomarbeitsteams problemlos Zugang zu ihrer eigenen
virtualisierten Maschine verschaffen. Die Virtualisierung dieses großen
Servercluster funktioniert mittels einer ProxMox-Umgebung.

\hypertarget{proxmox}{%
\subsubsection{ProxMox}\label{proxmox}}

Proxmox Virtual Environment (kurz PVE) ist eine auf Debian und KVM
basierende Virtualisierungs-Plattform zum Betrieb von
Gast-Betriebssystemen. Vorteile:

\begin{itemize}
\tightlist
\item
  läuft auf fast jeder x86-Hardware
\item
  frei verfügbar
\item
  ab 3 Servern Hochverfügbarkeit
\end{itemize}

Jedoch liegen alle Maschinen der Diplomarbeitsteams in einem eigens
gebaut und gesicherten Virtual Private Network (VPN), sodass nur mittels
eines eingerichteten Tools auf den virtualisierten Server zugegriffen
werden kann.

\hypertarget{forticlient}{%
\subsubsection{FortiClient}\label{forticlient}}

FortiClient ermöglicht es, eines VPN-Konnektivität anhand von IPsec oder
SSL zu erstellen. Die Datenübertragung wird verschlüsselt und damit der
enstandene Datenstrom vollständig gesichert über einen sogenannten
``Tunnel'' übertragen.

Da die Diplomarbeit ``Capentory'' jedoch Erreichbarkeit im Schulnetz
verlangt, muss die Maschine in einem Ausmaß abgesichert werden, damit
sie ohne Bedenken in das Schulnetz gehängt werden kann. Dafür müssen
folgende Punkte gewährleistet sein:

\begin{itemize}
\tightlist
\item
  Konfiguration beider Firewalls (siehe Punkt
  \ref{absicherung-der-virtuellen-maschine})
\item
  Wohlüberlegte Passwörter und Zugriffsrechte
\end{itemize}

\hypertarget{wahl-des-betriebssystems}{%
\subsection{Wahl des Betriebssystems}\label{wahl-des-betriebssystems}}

Neben den physischen Hardwarekomponenten wird für einen funktionierenden
und leicht bedienbaren Server logischerweise auch ein Betriebsystem
benötigt. Die erste Entscheidung, welche Art von Betriebssystem für die
Diplomarbeit ``Capentory'' in Frage kam wurde rasch beantwortet: Linux.
Weltweit basieren die meisten Server und andere Geräte auf Linux. Jedoch
gibt es selbst innerhalb des OpenSource-Hersteller zwei gängige
Distributionen, die das Diplomarbeitsteam während deren Schulzeit an der
Htl Rennweg kennenlernen und Übungen darauf durchführen durfte:

\begin{itemize}
\tightlist
\item
  Linux CentOS
\item
  Linux Ubuntu
\end{itemize}

\hypertarget{linux-centos}{%
\subsubsection{Linux CentOS}\label{linux-centos}}

CentOS ist eine frei verfügbare Linux Distribution, die auf
\href{https://de.wikipedia.org/wiki/Red_Hat_Enterprise_Linux}{Red Hat
Enterprise Linux} aufbaut. Hinter Ubuntu und Debian ist CentOS die am
dritthäufigsten verwendete Software und wird von einer offenen Gruppe
von freiwilligen Entwicklern betreut, gepflegt und weiterentwickelt.

\hypertarget{linux-ubuntu}{%
\subsubsection{Linux Ubuntu}\label{linux-ubuntu}}

Ubuntu ist die am meist verwendete Linux-Betriebssystemsoftware für
Webserver. Auf Debian basierend ist das Ziel der Entwickler, ein ein
einfach zu installierendes und leicht zu bedienendes Betriebssystem mit
aufeinander abgestimmter Software zu schaffen. Hauptsponsor des
Ubuntu-Projektes ist der Software-Hersteller
\href{https://de.wikipedia.org/wiki/Canonical}{Canonical}, der vom
südafrikanischen Unternehmer
\href{https://de.wikipedia.org/wiki/Mark_Shuttleworth}{Mark
Shuttlerworth} gegründet wurde.

\hypertarget{vergleich-und-wahl}{%
\subsubsection{Vergleich und Wahl}\label{vergleich-und-wahl}}

\textbf{CentOS}

\begin{itemize}
\tightlist
\item
  Kompliziertere Bedienung
\item
  Keine regelmäßigen Softwareupdates
\item
  Weniger Dokumentation vorhanden
\end{itemize}

\textbf{Ubuntu}

\begin{itemize}
\tightlist
\item
  Leichte Bedienung
\item
  Wird ständig weiterentwickelt und aktualisiert
\item
  Zahlreich brauchbare Dokumentation im Internet vorhanden
\item
  Wird speziell von Ralph empfohlen
\end{itemize}

Aus den angeführten Punkten entschied sich ``Capentory'' klarerweise das
Betriebsystem Ubuntu zu verwenden, vorallem auch weil der Hersteller
deren Serversoftware die Verwendung von diesem Betriebsystem empfiehlt.
Anschließend wird die Installation des Betriebsystemes genauer erläutert
und erklärt.

\hypertarget{installation-des-betriebsystemes}{%
\subsection{Installation des
Betriebsystemes}\label{installation-des-betriebsystemes}}

Wie bereits unter Punkt Anschaffung des Servers (siehe
\ref{anschaffung-des-servers}) erwähnt, wird uns von der Schule ein
eigener Servercluster mit virtuellen Maschinen zur Verfügung gestellt.
Durch die ProxMox-Umgebung und diversen Tools, ging die Installation der
Ubuntu-Distribution rasch von der Hand. In der Virtualisierungsumgebung
der Schule musste nur ein vorhandenes Linux-Ubuntu 18.04 ISO-File
gemountet und anschließend eine gewöhnliche Betriebsysteminstallation
für Ubuntu durchgeführt werden. Jedoch kam es beim ersten Versuch zu
Problemen mit der Konfiguration der Netzwerkschnittstellen, die im
nächsten Punkt genauer erläutert werden.

\hypertarget{konfiguration-der-netzwerkschnittstellen}{%
\subsection{Konfiguration der
Netzwerkschnittstellen}\label{konfiguration-der-netzwerkschnittstellen}}

\hypertarget{problematische-ereignise-bei-der-konfiguration-der-schnittstellen}{%
\subsubsection{Problematische Ereignise bei der Konfiguration der
Schnittstellen}\label{problematische-ereignise-bei-der-konfiguration-der-schnittstellen}}

Um eine funktiorierende Internetverbindung zu erstellen, durfte die
Konfiguration der Schnittstellen nicht erst ``später durchgeführt''
werden, da mit dem Aufschub der Schnittstellen-Konfiguration das Paket
``NetworkManager'' nicht installiert wurde. Der ``NetworkManager'' ist
verantwortlich für den Zugang zum Internet und der Netzwerksteuerung auf
dem Linuxsystem. Und da im Nachhinein dieses Paket nicht installiert war
(und aufgrund fehlender Internetverbindung nicht installiert werden
konnte), half auch die fehlerfreie Interface-Konfiguration nicht, um
eine Konnektivität herzustellen. Dadurch musste ein zweiter
Installationsdurchgang durchgeführt werden, worauf dann alles fehlerfrei
und problemlos lief.

\hypertarget{konfiguration}{%
\subsubsection{Konfiguration}\label{konfiguration}}

Im Rahmen des Laborunterrichts an der Htl Rennweg, bekamen die Schüler
für diverse Unklarheiten ein sogenanntes
\href{https://netzwerktechnik.htl.rennweg.at/~zai/Fachschule/NWT_Stamm_Foerderkurs/Survival_Guide_Linux_Network_Configuration.pdf}{Cheat-Sheet}
für Linux-Befehle zur Verfügung gestellt. In diesem Cheat-Sheet finden
sich unteranderem Anleitungen für die Konfiguration einer
Netzwerkschnittstelle auf einer CentOS/RedHat sowie
Ubuntu/Debian-Distribution. Den Schülern der fünften
Netzwerktechnikklasse sollte diese Kurzkonfiguration jedoch schon leicht
von der Hand gehen, da sie diese Woche für Woche benötigen.

Eine Netzwerkkonfiguration mit statischen IPv4-Adressen für eine
Ubuntu-Distribution könnte wie folgt aussehen:

In \texttt{/etc/network/interfaces}:

\begin{verbatim}
auto ens32
iface ens32 inet static
address 192.168.0.1
netmask 255.255.255.0
\end{verbatim}

Eine Netzwerkkonfiguration mit Verwendung eines IPv4-DHCP-Servers für
eine Ubuntu-Distribution könnte wie folgt aussehen:

In \texttt{/etc/network/interfaces}:

\begin{verbatim}
auto ens32
iface ens32 inet dhcp
\end{verbatim}

\hypertarget{topologie-des-netzwerkes}{%
\subsubsection{Topologie des
Netzwerkes}\label{topologie-des-netzwerkes}}

Unter Abbildung 5.1 wird der Netzwerkplan veranschaulicht. Auf der
linken Seite wird der Servercluster der Diplomarbeitsteams dargestellt,
worauf die virtuelle Maschine von ``Capentory'' gehostet wird. In der
Mitte ist die FortiGate-Firewall zu sehen, die nicht nur als äußerster
Schutz vor Angriffen dient, sondern auch die konfigurierte
VPN-Verbindung beinhaltet und nur berechtigten Teammitgliedern den
Zugriff gewährleistet. Desweiteren ist die moderne Firewall auch für die
Konnektivität der virtuellen Maschine im Schulnetz zuständig, aber dazu
später (unter Punkt ``Absicherung der virtuellen Maschine'') mehr.

\begin{figure}[ht]
\centering
\includegraphics{topo1.png}
\caption{Netzwerkplan}
\end{figure}

\hypertarget{testen-der-konnektivituxe4t-im-netzwerk}{%
\subsubsection{Testen der Konnektivität im
Netzwerk}\label{testen-der-konnektivituxe4t-im-netzwerk}}

\hypertarget{installation-der-notwendigen-applikationen}{%
\subsection{Installation der notwendigen
Applikationen}\label{installation-der-notwendigen-applikationen}}

Damit die Ubuntu-Maschine für den Produktivbetrieb startbereit ist,
müssen im Vorhinein noch einige wichtige Konfigurationen durchgeführt
werden. Die wichtigste (Netzwerkkonfiguration) wurde soeben ausführlich
erläutert doch ohne der Installation von diversen Applikationen, wäre
das System nicht brauchbar.

\hypertarget{advanced-packaging-tool}{%
\subsubsection{Advanced Packaging Tool}\label{advanced-packaging-tool}}

Jeder Ubuntu Benutzer kennt es. Mit diesem Tool werden auf dem System
die notwendigen Applikationen heruntergeladen, extrahiert und
anschließend installiert. Insgesamt stehen einem 18 apt-get commands zur
Verfügung. Genauere Erklärungen sowie die Syntax zu den wichtigsten
commands folgen.

\hypertarget{apt-get-update}{%
\paragraph{apt-get update}\label{apt-get-update}}

Update liest alle in \texttt{/etc/apt/sources.list} sowie in
\texttt{/etc/apt/sources.list.d/} eingetragenen Paketquellen neu ein.
Dieser Schritt wird vor allem vor einem upgrade-command oder nach dem
Hinzufügen einer neuen Quelle empfohlen, um sich die neusten
Informationen für Pakete ansehen zu können.

Syntax: \texttt{{[}sudo{]}\ apt-get\ {[}Option(en){]}\ update}

\hypertarget{apt-get-upgrade}{%
\paragraph{apt-get upgrade}\label{apt-get-upgrade}}

Upgrade bringt alle bereits installierten Pakete auf den neuesten Stand.

Syntax: \texttt{{[}sudo{]}\ apt-get\ {[}Option(en){]}\ upgrade}

\hypertarget{apt-get-install}{%
\paragraph{apt-get install}\label{apt-get-install}}

Install lädt das Paket bzw. die Pakete inklusive der noch nicht
installierten Abhängigkeiten (und eventuell der vorgeschlagenen weiteren
Pakete) herunter und installiert diese. Außerdem besteht die
Möglichkeit, beliebig viele Pakete auf einmal anzugeben, indem sie
mittels eines Leerzeichens getrennt werden.

Syntax:
\texttt{{[}sudo{]}\ apt-get\ {[}Option(en){]}\ install\ PAKET1\ {[}PAKET2{]}}

Falls eine bestimmte Version installiert werden soll:

\texttt{{[}sudo{]}\ apt-get\ {[}Option(en){]}\ install\ PAKET1=VERSION\ {[}PAKET2=VERSION{]}}

\hypertarget{apt-get-remove}{%
\paragraph{apt-get remove}\label{apt-get-remove}}

Wie bereits erkannt, gibt es die Möglichkeit, sich ein beliebiges Paket
zu installieren. Doch was soll geschehen falls dieses Paket nicht mehr
benötigt wird? Daher gibt es prakitscherweise den remove-command, der,
wie schon im Namen deutlich wird, ein oder mehrere Paket/e vollständig
vom System entfernt.

Syntax:
\texttt{sudo\ apt-get\ {[}Option(en){]}\ remove\ PAKET1\ {[}PAKET2{]}}

\hypertarget{installierte-pakete-mittels-apt-get}{%
\subsubsection{Installierte Pakete mittels
apt-get}\label{installierte-pakete-mittels-apt-get}}

\hypertarget{nginx}{%
\paragraph{NGINX}\label{nginx}}

NGINX ist der am Häufigsten verwendete OpenSource-Webserver unter Linux
für diverse Webanwendungen. Große Unternehmen wie Cisco, Microsoft,
Facebook oder auch IBM schwören auf die Verwendung dieses genialen
Paketes. Unteranderem wird NGINX auch als Reverse-Proxy, HTTP-Cache und
Load-Balancer verwendet. Wie genau NGINX für den Produktivbetrieb
funktioniert wird im Laufe des Punktes
\ref{produktivbetrieb-der-applikation} erläutert.

Installation: \texttt{{[}sudo{]}\ apt-get\ install\ nginx}

\hypertarget{docker}{%
\paragraph{Docker}\label{docker}}

Die OpenSource-Software Docker ist eine
Containervirtualisierungstechnologie, die die Erstellung und den Betrieb
von Linux Containern ermöglicht. Wie genau dies funktioniert, wird
später unter Punkt \ref{produktivbetrieb-der-applikation} genauer
beschrieben und erklärt.

Installation: \texttt{{[}sudo{]}\ apt-get\ install\ docker}

\hypertarget{docker-compose}{%
\paragraph{docker-compose}\label{docker-compose}}

Jeder Linux-Benutzer hat mindestens einmal in seinem Leben etwas über
das \texttt{docker-compose.yml} File gehört. Doch was ist docker-compose
eigentlich? Nun, die Verwaltung und Verlinkung von mehreren Containern
kann auf Dauer sehr nervenaufreibend sein. Die Lösung dieses Problems
nennt sich docker-compose. Wie docker-compose jedoch genau funktioniert,
wird ebenfalls wie das Grundkonzept von Docker unter Punkt
\ref{produktivbetrieb-der-applikation} präziser erläutert.

Installation: \texttt{{[}sudo{]}\ apt-get\ install\ docker-compose}

\hypertarget{mysql}{%
\paragraph{MySQL}\label{mysql}}

MySQL ist ein OpenSource-Datenverwaltungssystem und die Grundlage für
die meisten dynamischen Websiten. Darauf werden die Inventurdatensätze
der Htl Rennweg gespeichert. Nähere Informationen finden sich Punkt
\ref{produktivbetrieb-der-applikation} wieder.

Installation: \texttt{{[}sudo{]}\ apt-get\ install\ mysql}

\hypertarget{redis}{%
\paragraph{Redis}\label{redis}}

Redis ist eine In-Memory-Datenbank mit einer
Schlüssel-Wert-Datenstruktur (Key Value Store). Wie auch MySQL handelt
es sich um eine OpenSource-Datenbank. Warum Redis ebenfalls benötigt
wurde, wird ebenfalls später im Punkt
\ref{produktivbetrieb-der-applikation} erklärt.

Installation: \texttt{{[}sudo{]}\ apt-get\ install\ redis}

\hypertarget{virtualenv}{%
\paragraph{virtualenv}\label{virtualenv}}

Bei virtualenv handelt es sich um ein Tool, mit dem eine isolierte
Python-Umgebung erstellt werden kann. Eine solch isolierte Umgebung
besitzt eine eigene Installation von diversen Services und teilt ihre
libraries nicht mit anderen virtuellen Umgebungen (im optionalen Fall
greifen sie auch nicht auf die global installierten libraries zu). Dies
bringt vor allem den großen Vorteil, dass im Testfall virtuelle
Umgebungen aufgesetzt werden können, um nicht die globalen
Konfigurationen zu gefährden.

Installation: \texttt{{[}sudo{]}\ apt-get\ install\ virtualenv}

\hypertarget{python}{%
\paragraph{Python}\label{python}}

Python ist einer der Hauptbestandteile auf dem Serversystem der
Diplomarbeit ``Capentory''. Das Backend (=Serveranwendung) basiert wie
bereits erwähnt auf dem Python-Framework ``Django''. Um dieses Framework
auf dem System installieren zu können wird jedoch noch ein weiteres
``Packaging-Tool'', speziell für Python-Module, benötigt.

Installation: \texttt{{[}sudo{]}\ apt-get\ install\ python3.x}

\hypertarget{pip}{%
\subsubsection{pip}\label{pip}}

Und dieses Tool nennt sich Pip. Pip ist ein rekursives Akronym für
\textbf{P}ip \textbf{I}nstalls \textbf{P}ython und ist, wie bereits
erwähnt, das Standardverwaltungswerkzeug für Python-Module. Die Funktion
sowie Syntax kann relativ gut mit der von apt verglichen werden.

Installation von pip3 (vorrausgesetzt python3.x ist auf dem System
bereits installiert):

\texttt{{[}sudo{]}\ apt-get\ install\ python3-pip}

\hypertarget{installierte-pakete-mittels-pip3}{%
\subsubsection{Installierte Pakete mittels
pip3}\label{installierte-pakete-mittels-pip3}}

\hypertarget{uwsgi}{%
\paragraph{uWSGI}\label{uwsgi}}

Das eigentliche Paket, mit dem der Produktivbetrieb schlussendlich
gewährleistet wurde, nennt sich uWSGI. Speziell wurde es für die
Produktivbereitstellung von Serveranwendungen (wie eben der
Django-Server von Team ``Capentory'') entwickelt und harmoniert
eindrucksvoll mit der Webserver-Software NGINX. Die grundlegende
Funktionsweise von uWSGI, sowie eine Erklärung, warum schlussendlich
dieses Paket und nicht Docker verwendet wurde, wird unter Punkt
\ref{produktivbetrieb-der-applikation} veranschaulicht.

Installation: \texttt{{[}sudo{]}\ pip3\ install\ uwsgi}

\hypertarget{django}{%
\paragraph{Django}\label{django}}

Django ist ein in Python geschriebenes Webframework, auf dem unsere
Serveranwendung basiert. Genauere Informationen wurden jedoch schon
unter Punkt \ref{django-und-ralph} übermittelt.

Installation: \texttt{{[}sudo{]}\ pip3\ install\ Django}

\hypertarget{produktivbetrieb-der-applikation}{%
\subsection{Produktivbetrieb der
Applikation}\label{produktivbetrieb-der-applikation}}

\hypertarget{absicherung-der-virtuellen-maschine}{%
\subsection{Absicherung der virtuellen
Maschine}\label{absicherung-der-virtuellen-maschine}}

\hypertarget{uxfcberwachung-des-netzwerks}{%
\subsection{Überwachung des
Netzwerks}\label{uxfcberwachung-des-netzwerks}}

\hypertarget{verfassen-einer-serverdokumentation}{%
\subsection{Verfassen einer
Serverdokumentation}\label{verfassen-einer-serverdokumentation}}
