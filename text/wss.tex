Um allen Betroffenen einen problemlosen Produktivbetrieb zu
gewährleisten, muss ein physischer Server aufgesetzt werden. Auf diesem
können dann alle Komponenten unseres Git-Repositories geklont und
betriebsbereit installiert werden. Dafür gab es folgende Punkte zu
erfüllen:

\begin{itemize}
\tightlist
\item
  das Organisieren eines Servers
\item
  das Aufsetzen eines Betriebssystemes
\item
  die Konfiguration der notwendigen Applikationen
\item
  die Konfiguration der Netzwerkschnittstellen
\item
  das Testen der Konnektivität im Netzwerk
\item
  die Containervirtualisierung der einzelnen Komponenten (z.B.
  Datenbank, Webserver, \ldots)
\item
  das Verfassen einer Serverdokumentation
\item
  die Absicherung der Maschine
\item
  die Überwachung des Netzwerks
\end{itemize}

In den folgenden Kapiteln wird deutlich gemacht, wie die oben
angeführten Punkte, im Rahmen der vorliegenden Diplomarbeit abgearbeitet
wurden.

\hypertarget{anschaffung-des-servers}{%
\section{Anschaffung des Servers}\label{anschaffung-des-servers}}

Den 5. Klassen wird, dank gesponserter Infrastruktur, im Rahmen ihrer
Diplomarbeit ein Diplomarbeitscluster zur Verfügung gestellt. Damit
können sich alle Diplomarbeitsteams problemlos Zugang zu ihrer eigenen
virtualisierten Maschine verschaffen. Die Virtualisierung dieses großen
Servercluster funktioniert mittels einer ProxMox-Umgebung.

\hypertarget{proxmox}{%
\subsection{ProxMox}\label{proxmox}}

Proxmox Virtual Environment (kurz PVE) ist eine auf Debian und KVM
basierende Virtualisierungs-Plattform zum Betrieb von
Gast-Betriebssystemen. Vorteile: * läuft auf fast jeder x86-Hardware *
frei verfügbar * ab 3 Servern Hochverfügbarkeit

Jedoch liegen alle Maschinen der Diplomarbeitsteams aus
Sicherheitsgründen in einem gesicherten Virtual Private Network (VPN),
sodass nur mittels eines eingerichteten Tools auf den virtualisierten
Server zugegriffen werden kann.

\hypertarget{forticlient}{%
\subsection{FortiClient}\label{forticlient}}

FortiClient ermöglicht es, eine VPN-Konnektivität unter Einsatz von
IPsec oder SSL zu gewährleisten. Da die Datenübertragung verschlüsselt
wird, wird der enstandene Datenstrom vollständig gesichert über einen
sogenannten ``Tunnel'' übertragen.

Da die Diplomarbeit ``Capentory'' jedoch Erreichbarkeit im Schulnetz
verlangt, muss die Maschine in einem Ausmaß abgesichert werden, damit
sie ohne Bedenken in das Schulnetz gehängt werden kann. Dafür müssen
folgende Punkte gewährleistet sein: * Konfiguration beider Firewalls
(siehe Punkt \textbf{Serversicherung}) * Wohlüberlegte Passwörter und
Zugriffsrechte
