% Options for packages loaded elsewhere
\PassOptionsToPackage{unicode}{hyperref}
\PassOptionsToPackage{hyphens}{url}
%
\documentclass[
]{article}
\usepackage{lmodern}
\usepackage{amssymb,amsmath}
\usepackage{ifxetex,ifluatex}
\ifnum 0\ifxetex 1\fi\ifluatex 1\fi=0 % if pdftex
  \usepackage[T1]{fontenc}
  \usepackage[utf8]{inputenc}
  \usepackage{textcomp} % provide euro and other symbols
\else % if luatex or xetex
  \usepackage{unicode-math}
  \defaultfontfeatures{Scale=MatchLowercase}
  \defaultfontfeatures[\rmfamily]{Ligatures=TeX,Scale=1}
\fi
% Use upquote if available, for straight quotes in verbatim environments
\IfFileExists{upquote.sty}{\usepackage{upquote}}{}
\IfFileExists{microtype.sty}{% use microtype if available
  \usepackage[]{microtype}
  \UseMicrotypeSet[protrusion]{basicmath} % disable protrusion for tt fonts
}{}
\makeatletter
\@ifundefined{KOMAClassName}{% if non-KOMA class
  \IfFileExists{parskip.sty}{%
    \usepackage{parskip}
  }{% else
    \setlength{\parindent}{0pt}
    \setlength{\parskip}{6pt plus 2pt minus 1pt}}
}{% if KOMA class
  \KOMAoptions{parskip=half}}
\makeatother
\usepackage{xcolor}
\IfFileExists{xurl.sty}{\usepackage{xurl}}{} % add URL line breaks if available
\IfFileExists{bookmark.sty}{\usepackage{bookmark}}{\usepackage{hyperref}}
\hypersetup{
  hidelinks,
  pdfcreator={LaTeX via pandoc}}
\urlstyle{same} % disable monospaced font for URLs
\setlength{\emergencystretch}{3em} % prevent overfull lines
\providecommand{\tightlist}{%
  \setlength{\itemsep}{0pt}\setlength{\parskip}{0pt}}
\setcounter{secnumdepth}{-\maxdimen} % remove section numbering

\date{}

\begin{document}

\hypertarget{technische-umsetzung-infrastruktur}{%
\section{Technische Umsetzung:
Infrastruktur}\label{technische-umsetzung-infrastruktur}}

Um allen Kunden einen problemlosen Produktivbetrieb zu gewährleisten
muss ein physischer Server aufgesetzt werden. Auf diesem können dann
alle Komponenten unseres Git-Repositories geklont und betriebsbereit
installiert werden. Dafür gab es folgende Punkte zu erfüllen:

\begin{itemize}
\tightlist
\item
  das Organisieren eines Servers
\item
  das Aufsetzen eines Betriebssystemes
\item
  die Konfiguration der notwendigen Applikationen
\item
  die Konfiguration der Netzwerkschnittstellen
\item
  das Testen der Konnektivität im Netzwerk
\item
  die Einrichtung des Produktivbetriebes der Applikation
\item
  das Verfassen einer Serverdokumentation
\item
  die Absicherung der Maschine
\item
  die Überwachung des Netzwerks
\end{itemize}

In den folgenden Kapitel wird deutlich gemacht, wie die oben angeführten
Punkte, im Rahmen der Diplomarbeit ``Capentory'' abgearbeitet wurden.

\hypertarget{anschaffung-des-servers}{%
\subsection{Anschaffung des Servers}\label{anschaffung-des-servers}}

Den 5. Klassen wird, dank gesponserter Infrastruktur, im Rahmen ihrer
Diplomarbeit ein Diplomarbeitscluster zur Verfügung gestellt. Damit
können sich alle Diplomarbeitsteams problemlos Zugang zu ihrer eigenen
virtualisierten Maschine verschaffen. Die Virtualisierung dieses großen
Servercluster funktioniert mittels einer ProxMox-Umgebung.

\hypertarget{proxmox}{%
\subsubsection{ProxMox}\label{proxmox}}

Proxmox Virtual Environment (kurz PVE) ist eine auf Debian und KVM
basierende Virtualisierungs-Plattform zum Betrieb von
Gast-Betriebssystemen. Vorteile: 
\begin{itemize}
\tightlist
\item
  Läuft auf fast jeder x86-Hardware
\item
  Frei verfügbar
\item
  Ab 3 Servern Hochverfügbarkeit
\end{itemize}

Jedoch liegen alle Maschinen der Diplomarbeitsteams in einem eigens
gebaut und gesicherten Virtual Private Network (VPN), sodass nur mittels
eines eingerichteten Tools auf den virtualisierten Server zugegriffen
werden kann.

\hypertarget{forticlient}{%
\subsubsection{FortiClient}\label{forticlient}}

FortiClient ermöglicht es, eines VPN-Konnektivität anhand von IPsec oder
SSL zu erstellen. Die Datenübertragung wird verschlüsselt und damit der
enstandene Datenstrom vollständig gesichert über einen sogenannten
``Tunnel'' übertragen.

Da die Diplomarbeit ``Capentory'' jedoch Erreichbarkeit im Schulnetz
verlangt, muss die Maschine in einem Ausmaß abgesichert werden, damit
sie ohne Bedenken in das Schulnetz gehängt werden kann. Dafür müssen
folgende Punkte gewährleistet sein: 
\begin{itemize}
\tightlist
\item
  Konfiguration beider Firewalls (siehe Punkt \textbf{Absicherung der virtuellen Maschine})
\item
  Wohlüberlegte Passwörter und Zugriffsrechte
\item
  Weniger Dokumentation vorhanden
\end{itemize}

\hypertarget{wahl-des-betriebssystems}{%
\subsection{Wahl des Betriebssystems}\label{wahl-des-betriebssystems}}

Neben den physischen Hardwarekomponenten wird für einen funktionierenden
und leicht bedienbaren Server logischerweise auch ein Betriebsystem
benötigt. Die erste Entscheidung, welche Art von Betriebssystem für die
Diplomarbeit ``Capentory'' in Frage kam wurde rasch beantwortet: Linux.
Weltweit basieren die meisten Server und andere Geräte auf Linux. Jedoch
gibt es selbst innerhalb des OpenSource-Hersteller zwei gängige
Distributionen, die das Diplomarbeitsteam während deren Schulzeit an der
Htl Rennweg kennenlernen und Übungen darauf durchführen durfte:
\begin{itemize}
\tightlist
\item
  Linux CentOS
\item
  Linux Ubuntu
\end{itemize} 
\#\#\# Linux CentOS CentOS ist eine frei
verfügbare Linux Distribution, die auf
\href{https://de.wikipedia.org/wiki/Red_Hat_Enterprise_Linux}{Red Hat
Enterprise Linux} aufbaut. Hinter Ubuntu und Debian ist CentOS die am
dritthäufigsten verwendete Software und wird von einer offenen Gruppe
von freiwilligen Entwicklern betreut, gepflegt und weiterentwickelt.
\#\#\# Linux Ubuntu Ubuntu ist die am meist verwendete
Linux-Betriebssystemsoftware für Webserver. Auf Debian basierend ist das
Ziel der Entwickler, ein ein einfach zu installierendes und leicht zu
bedienendes Betriebssystem mit aufeinander abgestimmter Software zu
schaffen. Hauptsponsor des Ubuntu-Projektes ist der Software-Hersteller
\href{https://de.wikipedia.org/wiki/Canonical}{Canonical}, der vom
südafrikanischen Unternehmer
\href{https://de.wikipedia.org/wiki/Mark_Shuttleworth}{Mark
Shuttlerworth} gegründet wurde. \#\#\# Vergleich und Wahl
\textbf{CentOS} 
\begin{itemize}
\tightlist
\item
  Kompliziertere Bedienung
\item
  Keine regelmäßigen Softwareupdates
\item
  Weniger Dokumentation vorhanden
\end{itemize} 


\textbf{Ubuntu}

\begin{itemize}
\tightlist
\item
  Leichte Bedienung
\item
  Wird ständig weiterentwickelt und aktualisiert
\item
  Zahlreich brauchbare Dokumentation im Internet vorhanden
\item
  Wird speziell von Ralph empfohlen
\end{itemize}

Aus den angeführten Punkten entschied sich ``Capentory'' klarerweise das
Betriebsystem Ubuntu zu verwenden, vorallem auch weil der Hersteller
deren Serversoftware die Verwendung von diesem Betriebsystem empfiehlt.
Anschließend wird die Installation des Betriebsystemes genauer erläutert
und erklärt. \#\# Installation des Betriebsystemes Wie bereits unter
Punkt Anschaffung des Servers erwähnt, wird uns von der Schule
ein eigener Servercluster mit virtuellen Maschinen zur Verfügung
gestellt. Durch die ProxMox-Umgebung und diversen Tools, ging die
Installation der Ubuntu-Distribution rasch von der Hand. In der
Virtualisierungsumgebung der Schule musste nur ein vorhandenes
Linux-Ubuntu 18.04 ISO-File gemountet und anschließend eine gewöhnliche
Betriebsysteminstallation für Ubuntu durchgeführt werden. Jedoch kam es
beim ersten Versuch zu Problemen mit der ""``Konfiguration der
Netzwerkschnittstellen''"``, die im nächsten Punkt genauer erläutert
werden. \#\# Konfiguration der Netzwerkschnittstellen Um eine
funktiorierende Internetverbindung zu erstellen, durfte die
Konfiguration der Schnittstellen nicht erst''später durchgeführt"
werden, da mit dem Aufschub der Schnittstellen-Konfiguration das Paket
``NetworkManager'' nicht installiert wurde. Der ``NetworkManager'' ist
verantwortlich für den Zugang zum Internet und der Netzwerksteuerung auf
dem Linuxsystem. Und da im Nachhinein dieses Paket nicht installiert war
(und aufgrund fehlender Internetverbindung nicht installiert werden
konnte), half auch die fehlerfreie Interface-Konfiguration nicht, um
eine Konnektivität herzustellen. Dadurch musste ein zweiter
Installationsdurchgang durchgeführt werden, worauf dann alles fehlerfrei
und problemlos lief. \#\#\# Topologie des Netzwerkes Unter Abb.1 wird
der Netzwerkplan veranschaulicht.
\begin{figure}
	\includegraphics{topo1.png}
	\caption{Abb.1}
\end{figure}
\#\# Konfiguration der notwendigen
Applikationen

\hypertarget{testen-der-konnektivituxe4t-im-netzwerk}{%
\subsection{Testen der Konnektivität im
Netzwerk}\label{testen-der-konnektivituxe4t-im-netzwerk}}

\hypertarget{produktivbetrieb-der-applikation}{%
\subsection{Produktivbetrieb der
Applikation}\label{produktivbetrieb-der-applikation}}

\hypertarget{verfassen-einer-serverdokumentation}{%
\subsection{Verfassen einer
Serverdokumentation}\label{verfassen-einer-serverdokumentation}}

\hypertarget{absicherung-der-virtuellen-maschine}{%
\subsection{Absicherung der virtuellen
Maschine}\label{absicherung-der-virtuellen-maschine}}

\hypertarget{uxfcberwachung-des-netzwerks}{%
\subsection{Überwachung des
Netzwerks}\label{uxfcberwachung-des-netzwerks}}

\end{document}
