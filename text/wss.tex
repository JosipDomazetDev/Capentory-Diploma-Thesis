\hypertarget{technische-umsetzung-infrastruktur}{%
\section{Technische Umsetzung:
Infrastruktur}\label{technische-umsetzung-infrastruktur}}

Um allen Kunden einen problemlosen Produktivbetrieb zu gewährleisten
muss ein physischer Server aufgesetzt werden. Auf diesem können dann
alle Komponenten unseres Git-Repositories geklont und betriebsbereit
installiert werden. Dafür gab es folgende Punkte zu erfüllen:

\begin{itemize}
\tightlist
\item
  das Organisieren eines Servers
\item
  das Aufsetzen eines Betriebssystemes
\item
  die Konfiguration der notwendigen Applikationen
\item
  die Konfiguration der Netzwerkschnittstellen
\item
  das Testen der Konnektivität im Netzwerk
\item
  die Containervirtualisierung der einzelnen Komponenten (z.B.
  Datenbank, Webserver, \ldots)
\item
  das Verfassen einer Serverdokumentation
\item
  die Absicherung der Maschine
\item
  die Überwachung des Netzwerks
\end{itemize}

In den folgenden Kapitel wird deutlich gemacht, wie die oben angeführten
Punkte, im Rahmen der Diplomarbeit ``Capentory'' abgearbeitet wurden.

\hypertarget{anschaffung-des-servers}{%
\subsection{Anschaffung des Servers}\label{anschaffung-des-servers}}

Den 5. Klassen wird, dank gesponserter Infrastruktur, im Rahmen ihrer
Diplomarbeit ein Diplomarbeitscluster zur Verfügung gestellt. Damit
können sich alle Diplomarbeitsteams problemlos Zugang zu ihrer eigenen
virtualisierten Maschine verschaffen. Die Virtualisierung dieses großen
Servercluster funktioniert mittels einer ProxMox-Umgebung.

\hypertarget{proxmox}{%
\subsubsection{ProxMox}\label{proxmox}}

Proxmox Virtual Environment (kurz PVE) ist eine auf Debian und KVM
basierende Virtualisierungs-Plattform zum Betrieb von
Gast-Betriebssystemen. Vorteile: * läuft auf fast jeder x86-Hardware *
frei verfügbar * ab 3 Servern Hochverfügbarkeit

Jedoch liegen alle Maschinen der Diplomarbeitsteams in einem eigens
gebaut und gesicherten Virtual Private Networks (VPN), sodass nur
mittels eines eingerichteten Tools auf den virtualisierten Server
zugegriffen werden kann.

\hypertarget{forticlient}{%
\subsubsection{FortiClient}\label{forticlient}}

FortiClient ermöglicht es, eines VPN-Konnektivität anhand von IPsec oder
SSL zu erstellen. Die Datenübertragung wird verschlüsselt und damit der
enstandene Datenstrom vollständig gesichert über einen sogenannten
``Tunnel'' übertragen.

Da die Diplomarbeit ``Capentory'' jedoch Erreichbarkeit im Schulnetz
verlangt, muss die Maschine in einem Ausmaß abgesichert werden, damit
sie ohne Bedenken in das Schulnetz gehängt werden kann. Dafür müssen
folgende Punkte gewährleistet sein: * Konfiguration beider Firewalls
(siehe Punkt \textbf{Serversicherung}) * Wohlüberlegte Passwörter und
Zugriffsrechte

\hypertarget{das-aufsetzen-eines-betriebssystemes}{%
\subsection{Das Aufsetzen eines
Betriebssystemes}\label{das-aufsetzen-eines-betriebssystemes}}

\hypertarget{konfiguration-der-netzwerkschnittstellen}{%
\subsection{Konfiguration der
Netzwerkschnittstellen}\label{konfiguration-der-netzwerkschnittstellen}}

\hypertarget{konfiguration-der-notwendigen-applikationen}{%
\subsection{Konfiguration der notwendigen
Applikationen}\label{konfiguration-der-notwendigen-applikationen}}

\hypertarget{testen-der-konnektivituxe4t-im-netzwerk}{%
\subsection{Testen der Konnektivität im
Netzwerk}\label{testen-der-konnektivituxe4t-im-netzwerk}}

\hypertarget{containervirtualisierung-der-einzelnen-komponenten}{%
\subsection{Containervirtualisierung der einzelnen
Komponenten}\label{containervirtualisierung-der-einzelnen-komponenten}}

\hypertarget{verfassen-einer-serverdokumentation}{%
\subsection{Verfassen einer
Serverdokumentation}\label{verfassen-einer-serverdokumentation}}

\hypertarget{serversicherung}{%
\subsection{Serversicherung}\label{serversicherung}}

\hypertarget{uxfcberwachung-des-netzwerks}{%
\subsection{Überwachung des
Netzwerks}\label{uxfcberwachung-des-netzwerks}}
